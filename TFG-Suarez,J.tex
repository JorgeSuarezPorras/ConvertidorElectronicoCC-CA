\documentclass{report}
%% PAQUETES PARA FIGURAS
% REPORT

%\DeclareUnicodeCharacter{2264}
\DeclareUnicodeCharacter{2264}{\ensuremath{\leq}}
\DeclareUnicodeCharacter{0301}{\'{e}}
\DeclareUnicodeCharacter{0301}{\'{u}}
%% PAQUETES PARA FIGURAS
\usepackage{fancyhdr}
\usepackage[dvips]{epsfig}
\usepackage{epic}
\usepackage{eepic}
\usepackage{graphicx} 
\usepackage[rightcaption]{sidecap}
\usepackage{caption}
\usepackage{subcaption}

\usepackage{color} 
%\usepackage{subfigure}
\usepackage{rotating}
\usepackage{rotfloat}
%\floatstyle{boxed} 
\restylefloat{figure}
% \usepackage{float}      % Para usar [!h] en el posicionamiento de las tablas y figuras
\graphicspath{ {./Figuras/pdf/}{./Figuras/png/}{./Figuras/osciloscopio/} {./Figuras/fotografias/}} %Setting the graphicspath
% \DeclareGraphicsExtensions{.png,.pdf}
\usepackage{scrextend}
\usepackage{footnote}
\makesavenoteenv{tabular}
\makesavenoteenv{table}
%%%%% ESTO QUE SIGUE PARA PODER ESCRIBIR LOS ACENTOS DIRECTAMENTE
\usepackage[spanish]{babel}
\renewcommand\spanishtablename{\textbf{Tabla}}
\renewcommand\spanishfigurename{\textbf{Figura}}
\usepackage[bottom]{footmisc}
\usepackage[utf8]{inputenc}
\usepackage[T1]{fontenc}
% \usepackage[latin1]{inputenc}
\usepackage{hyphenat}
\usepackage{endnotes}
\usepackage{tabularx}
\usepackage{tabulary}

%% TOC HIPERLINKS
\usepackage{hyperref}
\hypersetup{
    colorlinks,
    citecolor=blue,
    filecolor=blue,
    linkcolor=blue,
    urlcolor=blue
}


% PARA PODER ESCRIBIR LO QUE YO QUIERA SIN QUE LO INTERPRETE
\usepackage{verbatim}
\usepackage{amsmath, amsthm, amssymb}

%%%% ESTO ES PARA LA BIBLIOGRAFIA 
\usepackage{csquotes}
\usepackage[backend=biber,style=ieee]{biblatex}  %style es el que quieres usar, est es el que mas se parece al que dicen las reglas
													   % Un estilo tipico es el ieee
													   % Hay mas estilos y otras cosas que se pueden poner 
													   % Usas el paquete biblatex y biber
\addbibresource{Bibliografia_V1.bib} % entre corchetes el fichero .bib Yo he puesto un ejemplo para que veas como se ponen las referencias

%% PDF VERSION

\pdfminorversion=7

% GEOMETRIA DEL DOCUMENTO
\usepackage{geometry}

\geometry{paperwidth=170mm,paperheight=240mm,textwidth=127mm, textheight=197mm,vmargin={20mm,24mm},hmargin={20mm,20mm},bindingoffset=3mm,includehead,headsep=7mm} % good

\setcounter{tocdepth}{1}

\begin{document}

\begin{titlepage}
    \centering
    {\includegraphics[width=0.4\textwidth]{logocomillas}\par}
    \vspace{0.5cm}
    {\bfseries\LARGE Grado en Ingeniería en Tecnologías Industriales \par}
    \vspace{2cm}
    {\scshape\LARGE Control de Corriente en un convertidor CC - CA mediante microcontrolador \par}
    \vspace{2cm}
    {\itshape\Large Trabajo Fin de Grado \par}
    \vspace{0.5cm}
    {\Large Autor: \par}
    {\Large Jorge Suárez Porras \par}
    \vspace{0.5cm}
    {\Large Director: \par}
    {\Large Aurelio García Cerrada \par}
    \vspace{0.5cm}
    {\Large Madrid \par}
    {\Large Enero 2020 \par}


\end{titlepage}

\newpage
\pagenumbering{gobble}
\section*{}
\vspace{8cm}
\begin{center}
    \begin{tabular}{@{}l@{}}
    \textit{Página en blanco.}
    \end{tabular}
    \end{center}

\newpage

\section*{}
\begin{flushright}
     \vspace{9cm}
\textit{A la religión, la vida y la ciencia, \\que todo lo abarcan y todo lo inundan.}
\end{flushright}

\newpage
\section*{}
\vspace{8cm}
\begin{center}
    \begin{tabular}{@{}l@{}}
    \textit{Página en blanco.}
    \end{tabular}
    \end{center}


\newpage
\pagenumbering{Roman}
\setcounter{page}{1}

\addcontentsline{toc}{section}{Resumen}
\chapter*{Resumen} \label{sec.resumen}
%%%%%%%% RESUMEN %%%%%%
El sistema energético actual de producción, distribución y consumo de electricidad procedente de centrales de eléctricas utiliza, principalmente, como materias primas aquellas procedentes de recursos no renovables. Dichas centrales producen y distribuyen la energía en corriente alterna a través de un complejo mallado que interconecta consumidores y productores. El control en la producción y la gestión de la energía en estos sistemas no renovables está bastamente extendido y es ampliamente conocido, por lo que su control es, hoy en día, seguro. No obstante, existe una fuerte tendencia hacia la obtención de energía mediante materias primas renovables, como son la energía solar, hidráulica, undimotriz, eólica... etc. Estas nuevas fuentes de energía se están implementando en forma de microrredes (del inglés \textit{micro-grids}) \cite{veinticinco}, de manera que pueden satisfacer ciertas demandas de energía eléctrica de manera local, por ejemplo, a complejos industriales, vecindarios residenciales, hospitales... etc \cite{veintidos}. Dado que se conectan de manera directa con la gran red de distribución, los modos de funcionamiento se restringen severamente en función del estado de la red. Es por ello, que en caso de que la red de distribución deje de suministrar potencia a los consumidores, estas fuentes renovables de energía se desconectan de la red por seguridad \cite{veintitres}, y no ayudan a paliar el incidente. 
Un mayor desarrollo de mejores tecnologías para estas conexiones con la red de distribución, unido a un desarrollo en el área de baterías y almacenamiento de energías, ayudarán significativamente a asentar con mayor seguridad fuentes renovables de energía en el sistema energético, haciendo que éste, de manera paulatina, evolucione a un modelo energético más ecológico, eficiente y seguro. 
Una de las principales tecnologías utilizadas para llevar a cabo una conexión de fuentes de energías renovables a la gran red de distribución, son los convertidores electrónicos de potencia CC - CA fuente de tensión (en inglés \textit{Voltage Source Converter} o \textit{VSC}). 

Dada la naturaleza del proyecto (destinado a docencia), las bajas tensiones que se operan y por facilidad de manejo y programación del control y su posterior implantación, se decidió utilizar un VSC que tiene como particularidad MOSFETS de GaN. El nitruro de galio o GaN se usa actualmente \cite{siete} para los interruptores de potencia destinados a ciertas aplicaciones \cite{veinticuatro}. Estos interruptores se enmarcan dentro de los llamados ``de banda prohibida'' o ``de banda ancha'' (en inglés \textit{wide-bandgap semiconductors} o \textit{WBG semiconductors}) y con ellos se pretende mejorar la conmutación (aumentándola), disminuir las pérdidas en conducción y conseguir tensiones de bloqueo más altas que las que se obtienen con los interruptores de Si (que esta limitada, teóricamente, a unos 6.5kV)\cite{cinco}.
En el presente documento, en primer lugar, se describirá cuál es el objetivo principal del proyecto. A continuación, se trata el aparato matemático que se utiliza para el control de los VSC. Seguidamente, se describirán los fundamentos de este tipo de convertidores; qué son, cómo se utilizan y controlan, de qué se componen... etc. Más adelante, se mostrará el convertidor eléctronico que se ha utilizado en el proyecto, así como el microcontrolador y los componentes que se han usado para construir el prototipo. Por último, se estudiarán una serie de controles, mostrando simulaciones en tiempos continuo y discreto, y su validación experimental en el prototipo.



\addcontentsline{toc}{section}{Palabras Clave}
\section*{Palabras clave} \label{sec.palabrasclave}
%%%%%%%%%%%%%%%%%%%%%%%%%%%%%%%%%%%%%%%%%%% PALABRAS CLAVE %%%%%%%%%%%%%%%%%%%%%%%%%%%%%%%%%%%%%%%%%%%
Inversor, VSC, Control de Corriente, Transformada de Park, LAUNCHPADXL - F28379D, TMS320F28379D, BOOSTXL - 3PhGaNInv.

%%%%%%% INDICE %%%%%%%%%%
\newpage

\tableofcontents
\listoffigures

\newpage
\pagenumbering{arabic} % para empezar la numeración con números
\setcounter{page}{1}
\chapter{Introducción} \label{sec.introduccion}

%%%%%%%%%%%%%%%%%%%%%%%%%%%%%%%%%%%%%%%%%%% INTRODUCCION %%%%%%%%%%%%%%%%%%%%%%%%%%%%%%%%%%%%%%%%%%%

El presente proyecto consiste en el desarrollo y control de un prototipo de un convertidor electrónico CC-CA fuente de tensión (en inglés \textit{Voltage Source Converter} o \textit{VSC}) trifásico para docencia. Tradicionalmente los VSCs se designaban en castellano como ``inversores'', porque la energía iba de la parte de continua a la parte de alterna, al contrario que en los rectificadores, no obstante esta nomenclatura resulta hoy en día un tanto obsoleta, ya que las nuevas tecnologías de los VSC permiten que sean dispositivos completamente reversibles. Este tipo de dispositivo tiene especial interés en campos como la inyección de potencia a la red eléctrica, proveniente por ejemplo de una granja solar, o bien para el control de motores eléctricos. 

Las tareas que se desarrollan serán:
\begin{enumerate}
    \item Elección de los componentes.
    \item Modelado del sistema que se quiere controlar.
    \item Simulación del sistema completo.
    \item Simulación en tiempo continuo con un regulador proporcional.
    \item Simulación en tiempo continuo con un regulador proporcional - integral.
    \item Simulación en tiempo discreto de los reguladores anteriores con el mayor número de detalles posibles que reflejen la implantación que se va a realizar.
    \item Implementación con los reguladores anteriores en una tarjeta de evaluación del microcontrolador TMS320F28739D.
    \item Prueba del prototipo y análisis de resultados.
\end{enumerate}


\section{Motivación del proyecto} \label{sec.motivaciondelproyecto}

La transición social hacia un desarrollo más sostenible, debe pasar necesariamente, por una mayor solidez en la implantación de las energías renovables en el sistema energético. Esta implantación debe hacerse, hoy en día, a través de convertidores electrónicos de potencia. Cualquier avance, análisis, diseño, prueba o desarrollo  que contribuya a la comprensión de estos sistemas facilitará dicha transición necesaria del modelo energético. La forma de redactar el proyecto se realizará siempre tratando de, como dice Carl Sagan en \textit{El mundo y sus demonios}\cite{cincuenta}, \enquote{hacerse accesible a la más amplia escala de comprensión fundamental de los descubrimientos y métodos de la ciencia}.

\section{Objetivos}\label{sec.objetivos}

El objetivo de este proyecto es realizar un control en lazo cerrado de la corriente de una bobina trifásica para ilustrar la conexión de un VSC (Voltage Source Converter) a la red eléctrica en la aplicación de las energías renovables. 
El resultado obtenido en el presente proyecto se pretende utilizar para docencia en diversas asignaturas relacionadas con la electrónica de potencia de grado y máster en la Universidad Pontificia Comillas.

\chapter{Convertidores fuente de tensión.} \label{sec.convertidoresfuentedetension}

% Se puede empezar a describir uno trifásico y nos saltamos el monofásico. Con onda cuadrada, se ve como el convertidor impone la tensión a la salida, si la tensión de entrada (CC) es constante (por eso el nombre de fuentes de tensión). PWM es la solución para tener tensión y frecuencia variable: breve descripción de los conceptos de PWM.
% Descripción del prototipo de inversor y un poco sobre los semiconductores.
Los convertidores fuentes de tensión (en inglés Voltage Source Converter o VSC) son dispositivos electrónicos de potencia, los cuales se basan en la conmutación de dispositivos semiconductores de potencia, que transforman una tensión continua de entrada en una tensión alterna de salida.

Los convertidores electrónicos de potencia fuente de tensión, permiten, si a la entrada se tiene una tensión variable, a la salida puede obtenerse, mediante la ganancia del sistema, una tensión variable. Sin embargo, si a la entrada se tiene una tensión constante, a la salida puede también obtenerse una tensión variable si se controla el VSC mediante modulación de ancho de pulso \cite{dieciocho}, siendo este último caso un convertidor CC - CA como el desarrollado en el presente proyecto.

El esquema básico del sistema se muestra en la Fig. \ref{fig.inversortrifasicoesquematico}.

\begin{figure}[!h]
    \begin{center}
    \resizebox{8cm}{!}{\includegraphics{inversortrifasicoesquematico}}
    \end{center}
    \caption{Esquema de un VSC trifásico. Vdc es la tensión continua de entrada, Va0, Vb0 y Vc0 son las tensiones alternas de salida.}
    \label{fig.inversortrifasicoesquematico}
    \end{figure}

\section{Breve descripcion del PWM.}\label{sec.brevedescripcionpwm}

La técnica de control de los VSC mediante modulación de ancho de pulso (del inglés Pulse Width Modulation o PWM), permite obtener una tensión cuyo armónico fundamental será de amplitud variable (alterna), a partir de una tensión de entrada fija (continua). Una salida PWM consiste en pulsos de frecuencia de conmutación y magnitud fija, pero ancho de pulso variable \cite{veintiseis}. No obstante existen casos en los que la frecuencia de conmutación puede ser variable, como en la mitigación del ruido audible en determinadas aplicaciones de electrónica de potencia. Una salida PWM puede generarse mediante dos técnicas principalmente, comparación de una senoidal con una triangular, como en la Fig. \ref{fig.triangular_senoidal} y mediante PWM en vectores espaciales (del inglés \textit{Space Vector based on PWM} o \textit{SVPWM}), representado en la Fig. \ref{fig.SVPWM}. Ésta última técnica, SVPWM, se adapta a la conmutación de los transistores de potencia (ver Sección \ref{sec.semiconductores}), en función de la carga a la que se conecta el VSC \cite{veintiocho}.

\begin{figure}[!h]
    \begin{center}
    \resizebox{8cm}{!}{\includegraphics{triangular_senoidal}}
    \end{center}
    \caption{Generación de PWM mediante comparación con una triangular \cite{veintisiete}}
    \label{fig.triangular_senoidal}
    \end{figure}

\begin{figure}[!h]
        \begin{center}
        \resizebox{7cm}{!}{\includegraphics{SVPWM}}
        \end{center}
        \caption{Tensiones en Vectores Espaciales \cite{veintiocho}}
        \label{fig.SVPWM}
        \end{figure}   

\clearpage
\section{Transistores de Potencia} \label{sec.semiconductores}

Los transistores de potencia son interruptores que pueden activarse (cortocircuito) o desactivarse (circuito abierto) usando una señal de control. Hoy en día, dada la gran evolución de estos dispositivos en la industria, permiten conmutar entre los estados activo e inactivo a velocidades mucho mayores que antes, lo que facilita la incorporación de los transistores en convertidores electrónicos de potencia CC - CA, así como CA - CC \cite{dieciocho}. Se describen a continuación tres de los grandes grupos en los que puede dividirse el basto mundo de los transistores de potencia. 

\subsection{Transistor tipo MOSFET} \label{sec.mosfet}
Los transistores tipo MOSFET (del inglés \textit{Metal Oxide Semiconductor Field Effect Transistor}) son interruptores controlados en tensión, requeriendo una corriente de entrada en el terminal de control (G, en la Fig. \ref{fig.diagrama_mosfets}) solo en los primeros instantes de la conmutación \cite{dieciocho}. Permiten frecuencias de conmutación muy elevadas \cite{treinta}, facilitando así su inmersión en los convertidores electrónicos de potencia. En la Fig. \ref{fig.diagrama_mosfets} \cite{uno} puede verse un diagrama de un MOSFET en configuración canal tipo n (Fig. \ref{fig.diagrama_mosfets}-a) y configuración canal tipo p (Fig. \ref{fig.diagrama_mosfets}-b)
\begin{figure}[!h]
    \begin{center}
    \resizebox{8cm}{!}{\includegraphics{diagrama_mosfets}}
    \end{center}
    \caption{Diagrama de un transistor MOSFET a) canal n y b) canal p. \cite{uno}}
    \label{fig.diagrama_mosfets}
    \end{figure}

En caso de que el sustrato no se encuentre accesible (y esto ocurre siempre en electrónica de potencia), la Fig. \ref{fig.diagrama_mosfets_enriquecimiento_empobrecimiento} muestra su simbología.

\begin{figure}[!h]
    \begin{center}
    \resizebox{8cm}{!}{\includegraphics{diagrama_mosfets_enriquecimiento_empobrecimiento}}
    \end{center}
    \caption{Diagrama de un transistor MOSFET canal n y (b) Diagrama de un transistor MOSFET canal p. \cite{uno}}
    \label{fig.diagrama_mosfets_enriquecimiento_empobrecimiento}
    \end{figure}

\subsection{Transistor tipo BJT} \label{sec.bjt}
Los transistores bipolares de juntura o BTJ (del inglés \textit{Bipolar Junction Transistor}) son dispositivos que se controlan por corriente. Éstos requieren de un flujo de corriente por la base (terminal de control, B en la Fig. \ref{fig.diagrama_bjt}), para que, al crear una diferencia de potencial entre la base y el emisor, se permita el flujo de corriente por el colector. En estos dispositivos, la ganancia de corriente es altamente dependiente de la temperatura \cite{diecinueve}. En la Fig. \ref{fig.diagrama_bjt} \cite{veintiuno} se muestra un transistor de este tipo en conducción directa. Un transistor BJT necesita corriente en B mientras que tenga que estar activo (conduciendo).

\begin{figure}[!h]
    \begin{center}
    \resizebox{8cm}{!}{\includegraphics{diagrama_bjt}}
    \end{center}
    \caption{Diagrama de un transistor BJT en conducción directa \cite{veintiuno}}
    \label{fig.diagrama_bjt}
    \end{figure}

    Los BJT fueron populares en convertidores CC-CA en la década de 1975-1985, pero han sido sustituidos por MOSFETs en aplicaciones de baja tensión y baja-media potencia o por IGBTs en aplicaciones de media-alta tensión y media-alta potencia, o por otros interruptores en aplicaciones de muy alta tensión y potencia \cite{dieciocho}.


\subsection{Transistor tipo IGBT} \label{sec.igbt}

Los transistores de potencia IGBT también permiten la conmutación entre estados activo e inactivo a alta frecuencia \cite{veinte}, pero la frecuencia de conmutación máxima es sensiblemente menor en los IGBTs que en los MOSFETs. Un transistor IGBT tiene una alta impedancia de entrada y bajas pérdidas en conducción en activo \cite{diecinueve}. Al comportarse de manera similar a una combinación entre un BJT y un MOSFET, su representación puede realizarse como BJT (Fig. \ref{fig.igbt}-a) o como MOSFET (Fig. \ref{fig.igbt}-b).

\begin{figure}[!h]
    \begin{center}
    \resizebox{7cm}{!}{\includegraphics{igbt}}
    \end{center}
    \caption{Diagrama de un IGBT como a) BJT b) MOSFET}
    \label{fig.igbt}
    \end{figure}

    \subsection{Materiales de los semiconductores} \label{sec.materialessemiconductores}

    Los materiales con los que se fabrican los transistores de potencia han sufrido grandes evoluciones en los últimos años gracias al desarrollo e implantación de la electrónica de potencia en campos como el almacenamiento de energía, el auge de los vehículos eléctricos, el desarrollo de las llamadas microrredes (del inglés \textit{micro grids})... etc. 
    Típicamente, los materiales semiconductores más utilizados son el germanio (Ge), el silicio (Si) y el arseniuro de galio (GaAs). Existen, dentro de estos, materiales intrínsecos y extrínsecos. Un material intrínseco es aquel que ha sido refinado de manera que presenta un bajo índice de impurezas, mientras que un material extrínseco es aquél que ha sido dopado con impurezas. Dentro de estos últimos, existen materiales extrínsecos de canal n y de canal p. Un material extrínseco tipo n se forma añadiendo átomos que tienen cinco electrones de valencia (arsénico, fósforo...), mientras que un material extrínseco de canal p se forma añadiendo átomos que tienen tres electrones de valencia (boro, galio, indio... ). En cualquiera de los tipos, la activación o desactivación del material semiconductor se produce debido a flujos de electrones o huecos de éstos \cite{uno}.
    
    Los materiales semiconductores pueden ser controlados para conmutar entre modo activo o inactivo. La transición entre estos dos niveles se lleva a cabo cuando se le aplica suficiente energía a los electrónes para poder superar la llamada \textit{banda prohibida} (del inglés \textit{bandgap}), como puede observarse en la Fig. \ref{fig.bandadeconduccion}. 
    \begin{figure}[!h]
        \begin{center}
        \resizebox{12cm}{!}{\includegraphics{bandadeconduccion}}
        \end{center}
        \caption{Banda de conducción de materiales semiconductores\cite{uno}.}
        \label{fig.bandadeconduccion}
        \end{figure}
    
    Hoy en día se esta utilizando el nitruro de galio (GaN) o el carburo de silicio (SiC) para construir interruptores con ventajas frente a los de silicio: menos pérdidas en conmutación, menor resistencia de canal en MOSFETs, mayores tensiones de bloqueo...etc. \cite{veintinueve}.
    
\subsection{Tiempo Muerto} \label{sec.tiempomuerto}
Para que no se produzca un cortocircuito en la fuente de tensión del convertidor electrónico, es necesario que los interruptores de la misma rama cambien su estado de $t_{on}$ a $t_{off}$ y de $t_{off}$ a $t_{on}$ de manera que no coincidan en modo activo al mismo tiempo. Véase la Fig. \ref{fig.cortocircuitodefases}. Los semiconductores, por norma general, pueden pasar de un estado $off$ a un estado $on$ en un periodo muy corto de tiempo. Sin embargo, el tiempo que tardan los semiconductores en pasar de un estado en $on$ a un estado en $off$, es considerablemente mayor \cite{uno}, Fig. \ref{fig.deadtime}.

\begin{figure}[ht]
    \begin{center}
    \resizebox{8cm}{!}{\includegraphics{cortocircuitodefases}}
    \end{center}
    \caption{Hipotético cortocircuito de la fase A}
    \label{fig.cortocircuitodefases}
    \end{figure}
    
Por lo tanto, dado que los interruptores no se apagan ni encienden de forma instantánea, se debe retrasar el encendido de cada interruptor de cada rama, de manera que el semiconductor complementario tenga tiempo de apagarse completamente, como puede observarse en la Fig. \ref{fig.cortocircuitodefasesdeadtime}. Así se evita que dos interruptores de la misma rama conduzcan simultaneamente. Este efecto recibe el nombre de \textit{tiempo muerto} (en inglés, \textit{dead time}).

\begin{figure}[!h]
\begin{center}
\resizebox{8cm}{!}{\includegraphics{deadtime}}
\end{center}
\caption{Tiempo de encendido y apagado de un semiconductor} \label{fig.deadtime}
\end{figure}

\begin{figure}[!h]
\begin{center}
\resizebox{8cm}{!}{\includegraphics{cortocircuitodefasesdeadtime}}
\end{center}
\caption{Tiempo en el que una rama se cortocircuita.}
\label{fig.cortocircuitodefasesdeadtime}
\end{figure}



\subsection{Método analógico} \label{sec.metodoanalogico}
Analógicamente, el desfase necesario para abarcar el Dead Time descrito en la Sección \ref{sec.tiempomuerto}, puede realizarse mediante un circuito analógico sencillo como se muestra en la Fig. \ref{fig.metodoanalogico}.

\begin{figure}[!h]
\begin{center}
\resizebox{6cm}{!}{\includegraphics{metodoanalogico}}
\end{center}
\caption{Circuito analógico para satisfacer el Dead Time.}
\label{fig.metodoanalogico}
\end{figure}

\subsection{Método digital} \label{sec.metododigital}
Dependiendo de la arquitectura y el lenguaje en el que se programan los cálculos para el disparo de los semiconductores, la metodología puede variar. No obstante, consiste típicamente en retrasar el encendido del semiconductor complementario de una fase, de manera que en ningún caso se cortocircuite la fuente de alimentación.




\section{VSCs con modulación de ancho de pulso} \label{sec.controldesemiconductoresmediantePWM}

Los MOSFETs, como se describe en la Sección \ref{sec.semiconductores}, se controlan mediante una tensión entre puerta y fuence ($Gate$ y $Source$). Una vez se activa la conducción por el MOSFET, no es necesario seguir suministrándole corriente. 

Estas tensiones tendrán una forma de onda cuadrada cuyo factor de servicio (del inglés \textit{Duty Cycle}) se define como el tiempo en el que la onda se encuentra a nivel alto ($t_{on}$) divido entre el periodo total de la onda ($T$), como puede verse en (\ref{eq.factordeservicio}) y en la Fig. \ref{fig.factordeservicio}.
Dado que a la salida quiere tenerse una forma de onda senoidal, los factores de servicio deberán variar senoidalmente.

Estos pulsos que encienden y apagan los semiconductores se realizan individualmente para cada fase, y en concreto, para cada semiconductor. Dado que se pretende realizar un VSC con la salida de alterna en trifásica, se computarán y producirán un total de 6 señales de PWM distintas, es decir, seis ondas diferentes. No obstante, existe cierta relación entre los semiconductores que actúan sobre una fase. 
Esto se debe a que dos semiconductores de la misma rama no pueden estar encendidos al mismo tiempo, para evitar un cortocircuito en la fuente de tensión continua. Más detalles en la Sección \ref{sec.tiempomuerto}.



\begin{figure}[!h]
    \begin{center}
    \resizebox{8cm}{!}{\includegraphics{factordeservicio}}
    \end{center}
    \caption{Definición del factor de servicio.}
    \label{fig.factordeservicio}
    \end{figure}
    
    %ECUACIONES FACTOR DE SERVIVIO:
    \begin{eqnarray}
    D = t_{on} / T \label{eq.factordeservicio}
    \end{eqnarray} 
    Con:
    \begin{eqnarray}
    T = t_{on} + t_{off} 
    \label{eq.definiciondeperiodo}
    \end{eqnarray}



\section{Descripción de un VSC} 

Un VSC fuente de tensión CC-CA consiste en una fuente de tensión continua gobernable, una serie de interruptores controlados mediante un microprocesador, una salida de tensión alterna que surge de la conmutación de los interruptores y una carga a la que se conecta (En el presente proyecto es una bobina trifásica.) 

El sistema de control del sistema, consiste en \cite{treintayuno}: 
\begin{enumerate}
    \item Se coloca la posición inicial de las salidas de los contadores para PWM.
    \item Se disparan los ADC y se espera a que acaben de convertir.
    \item Se calcula la tensión que hay que aplicar a la carga.
    \item Se convierte dicho valor de tensión en un factor de servicio para cada rama.
    \item Se actualiza la generación de PWM (si procede).
\end{enumerate} 

 Un modelo de VSC alimentando una carga, se ilustra en la Fig. \ref{fig.inversortrifasicoesquematicofiltroRL_2}.  Mientras que en la Fig. \ref{fig.diagramaVSCmarcado} se muestra el diagrama funcional del VSC utilizado en el prototipo.

En el caso del presente prototipo, dado que consiste en un control de corriente, se muestrean las intensidades que circulan por la carga (ia, ib e ic en Fig. \ref{fig.inversortrifasicoesquematicofiltroRL_2}), utilizadas para controlar la tensión que alimenta la carga y así seguir la referencia impuesta en el control. 


\begin{figure}[!h]
    \begin{center}
    \resizebox{12cm}{!}{\includegraphics{inversortrifasicoesquematicofiltroRL_2}}
    \end{center}
    \caption{Esquema de un VSC alimentando una carga. Vdc es la tensión de continua de entrada gobernable. T1 a T6 son los interruptores disparados con el microprocesador. Va0, Vb0 y Vc0 son las tensiones alternas de salida e ia, ib e ic las corrrientes por la carga. L y R es la carga que se alimenta.}
    \label{fig.inversortrifasicoesquematicofiltroRL_2}
    \end{figure}



    \begin{figure}[!h]
        \begin{center}
        \resizebox{12cm}{!}{\includegraphics{diagramaVSCmarcado}}
        \end{center}
        \caption{Diagrama funcional del VSC utilizado en el prototipo.}
        \label{fig.diagramaVSCmarcado}
        \end{figure}


\subsection{Muestreo de las señales}\label{sec.muestreodesenales} 

El ADC de los microcontroladores muestrea la señal de salida (procedente de las sondas de tensión y corriente) entre un intante $k$ y un instante consecutivo $k+1$. Será estrictamente necesario que los ADC comiencen su muestreo en $k$ y es condición indispensable que los niveles de los PWM de las señales de las tres fases del inversor, estén simultaneamente a nivel alto o nivel bajo. Como puede observarse en la Fig. \ref{fig.ADC_PWM_2}.

\begin{figure}[!h]
\begin{center}
\resizebox{9cm}{!}{\includegraphics{pwm-adc_2}}
\end{center}
\caption{Sincronización del ADC con PWM con un único disparo por ciclo.}
\label{fig.ADC_PWM_2}
\end{figure}


Así mismo, para conseguir la condición anterior, se pueden utilizar distintos métodos. El método que se va a emplear en el prototipo, es el de dividir el pulso a nivel alto del PWM de cada ciclo en dos partes iguales, es decir, conseguir el mismo factor de servicio $D$ en un ciclo de periodo $T$ (no necesariamente igual al tiempo comprendido entre $k$ y $k+1$), mediante la división en dos pulsos de valor $D/2$ para el mismo periodo $T$, como se muestra en Fig. \ref{fig.ADC_PWM_3}. De esta manera, se consigue con mayor seguridad que los tres PWM se encuentren a nivel alto cuando convierte el ADC.


\begin{figure}[!h]
\begin{center}
\resizebox{9cm}{!}{\includegraphics{pwm-adc_3}}
\end{center}
\caption{Sincronización del ADC con PWM con doble disparo por ciclo.}
\label{fig.ADC_PWM_3}
\end{figure}


Dada la gran potencia de los microcontroladores actuales, pueden realizarse los cálculos necesarios para hallar el valor del factor de servicio de los PWM para obtener como resultado la variable de control, en un breve periodo de tiempo. Dicho periodo de tiempo, estará comprendido entre $k$ y $k+1$. Es por ello, que ha de analizarse la necesidad o no de ejecutar la orden de actualizar el valor del nuevo factor de servicio antes de que acabe el ciclo entre $k$ y $k+1$, o en su defecto, esperar a que se complete el ciclo y actualizar el valor en $k+1$. Dada la extensión y objetivo de este proyecto, se opta por la segunda opción: esperar a que se complete el ciclo.
\clearpage

\chapter{Conexión de convertidores electrónicos de tensión a la red eléctrica} \label{sec.conexiondeconvertidoresaredelectrica}

% Describir algunas aplicaciones: eólica, solar, por ejemplo. Mencionar la necesidad de un filtro y poner ejemplos de filtros que se usan: L, LC, LCL (y los hay más complicados). Hay dos tipos de conexión que se llaman: “grid forming” y “grid supporting” (voy a mandarte un artículo donde se incluyen). Se pueden describir brevemente. Aclarar que nos vamos a decantar por estudiar el control por un filtro L, solamente y, como ejercicio … no lo vamos a conectar a la red. Diseño del sistema de control.


Los convertidores electrónicos CC - CA pueden conectarse a la red eléctrica de tres maneras, en formato \textit{grid-forming}, en formato \textit{grid-supporting} y en formato \textit{grid-feeding} \cite{dos}.

Los VSC utilizados en formato \textit{grid-forming} pueden asemejarse a un modelo de fuente de tensión alterna ideal, con una amplitud y una frecuencia constante, teniendo una impedancia de salida muy pequeña, como se muestra en la Fig. \ref{fig.grid-forming-diagram}. La amplitud y la frecuencia de la tensión viene determinada por la red local a la que quiere conectarse \cite{dos} o en su defecto, en funcionamiento en modo aislado \cite{treintaysiete}, debe seguir una referencia fija \cite{ocho}
    

\begin{figure}[!h]
    \begin{center}
    \resizebox{8cm}{!}{\includegraphics{grid-forming-diagram}}
    \end{center}
    \caption{Diagrama de un convertidor en formato \textit{grid-forming}\cite{nueve}.}
    \label{fig.grid-forming-diagram}
    \end{figure}

El formato \textit{grid-feeding} se utiliza para poder controlar las potencias que circulan por la red, consumiendo o generando potencias activa o reactiva, de manera fija o variable \cite{dos}. Debe estar perfectamente sincronizada con la red local a la que quiere conectarse \cite{ocho}. Se modelan como una fuente de intensidad ideal conectada en paralelo con una gran impedancia, Fig. \ref{fig.grid-feeding-diagram}. 


\begin{figure}[!h]
    \begin{center}
    \resizebox{8cm}{!}{\includegraphics{grid-feeding-diagram}}
    \end{center}
    \caption{Diagrama de un convertidor en formato \textit{grid-feeding}\cite{nueve}.}
    \label{fig.grid-feeding-diagram}
    \end{figure}

Por último, el formato \textit{grid-supporting} puede controlar la frecuencia y amplitud de la tensión de salida, mediante un modelo basado en una fuente de tensión de alterna en serie con una impedancia, como se muestra en la Fig. \ref{fig.grid-forming-supporting-diagram}, o bien controlar las potencias de la red mediante el modelado de una fuente de corriente en paralelo con una carga, como se muestra en la Fig. \ref{fig.grid-feeding-supporting-diagram}. El formato \textit{grid-supporting} consiste en actuar, bien sobre las potencias inyectadas o consumidas a partir de las medidas de tensión y frecuencia de la red (\textit{grid-supporting-feeding}), o bien actuar sobre la tensión y frecuencia a partir de las medidas de potencias activa y reactiva de la red (\textit{grid-supporting-forming}). \cite{seis}. La potencia activa esta fuertemente relacionada con la frecuencia de la red, mientras que la potencia reactiva se relaciona con el perfil de tensiones \cite{cuarentaycinco}.



\begin{figure}[!h]
    \begin{center}
    \resizebox{8cm}{!}{\includegraphics{grid-forming-supporting-diagram}}
    \end{center}
    \caption{Diagrama de un convertidor en formato \textit{grid-supporting-forming} con modelo de fuente de tensión \cite{seis}.}
    \label{fig.grid-forming-supporting-diagram}
    \end{figure}
    
\begin{figure}[!h]
    \begin{center}
    \resizebox{8cm}{!}{\includegraphics{grid-feeding-supporting-diagram}}
    \end{center}
    \caption{Diagrama de un convertidor en formato \textit{grid-supporting-feeding} con modelo de fuente de corriente \cite{seis}.}
    \label{fig.grid-feeding-supporting-diagram}
    \end{figure}

\clearpage
\section{Sincronización con la red} \label{sec.sincronizacionconlared}

La sincronización de los convertidores eléctronicos CC - CA con la red es fundamental para su correcto funcionamiento. Esta sincronización depende en primera instancia del formato de funcionamiento del VSC. Cuando el VSC se encuentra conectado a la red, la frecuencia de funcionamiento viene determinada por la frecuencia de la red, en cambio, funcionando en modo isla o aislado, la frecuencia de funcionamiento será una referencia impuesta por el operador del VSC. En caso de estar conectado a la red, la sincronización se realiza a través de un bloque de sincronización \textit{PLL} \cite{dos} (del inglés \textit{Phase-Locked Loop}), ver Fig. \ref{fig.PLL}. Este bloque consiste en un seguidor de referencia senoidal, que reproduce la frecuencia de la red eléctrica a la salida, en la Fig. \ref{fig.PLL}, el ángulo $\theta$. Se puede comprobar en la Fig. \ref{fig.grid-forming-PLL} cómo se utilizaría un \textit{PLL} en un VSC funcionando en formato \textit{grid-forming} \cite{doce,treintaysiete}.


\begin{figure}[!h]
    \begin{center}
    \resizebox{8cm}{!}{\includegraphics{PLL}}
    \end{center}
    \caption{Diagrama de un bloque \textit{PLL} \cite{once}.}
    \label{fig.PLL}
    \end{figure}


\begin{figure}[!h]
    \begin{center}
    \resizebox{8cm}{!}{\includegraphics{grid-forming-PLL}}
        \end{center}
        \caption{Modelo linealizado de un \textit{PLL} \cite{doce}.}
        \label{fig.grid-forming-PLL}
        \end{figure}

    
En el caso de querer sincronizar una red alterna trifásica, se utiliza el bloque \textit{SRF-PLL} \cite{diez} (del inglés \textit{Synchronous Reference Frame Phase-Locked Loop}), ver Fig. \ref{fig.SRF-PLL}) 

\begin{figure}[!h]
    \begin{center}
    \resizebox{8cm}{!}{\includegraphics{SRF-PLL}}
    \end{center}
    \caption{Modelo linealizado de un \textit{SRF-PLL} \cite{diez}.}
    \label{fig.SRF-PLL}
    \end{figure}



\clearpage
\section{Algunos filtros simples} \label{sec.tiposdefiltros}
Hoy en día existe una fuerte tendencia en la aplicación de los VSC a la hora de integrar fuentes de energía renovables al sistema eléctrico, el cuál opera en magnitudes alternas. Dicha conexión a la red eléctrica, requiere de la implantación de filtros para poder generar ondas trifásicas que se sincronicen con la red y que supriman los armónicos \cite{treintaycuatro}. Los filtros van desde los más sencillos como los filtros en L, LC y LCL \cite{treintaycinco,treintaysiete,treintayocho}, hasta llegar a filtros de mucha mayor complejidad. En el presente prototipo se utilizará un filtro en L. Además, será meramente experimental, sin conectar el convertidor a la red eléctrica.

No obstante, la utilización de los VSC también está extendido al control de motores eléctricos, que, debido a su construcción, son cargas fuertemente inductivas. Esto hace que no sea necesaria la implantación de filtros a la salida del VSC en muchas ocasiones.

 El modelado de la red eléctrica se realiza suponiendo que ésta es equivalente a una carga trifásica. La aplicación de conectar el VSC a la red sí requiere la utilización de filtros, siendo los que se enumeran a continuación los más sencillos,

\begin{enumerate}
    \item Filtro L. En la alimentación de una carga pasiva con un filtro L como el de la Fig. \ref{fig.filtroL} a la salida del inversor, se controlará la corriente por cada una de las fases.
    
    \item Filtro LC. En la alimentación de una carga pasiva con un filtro LC como el de la Fig. \ref{fig.filtroLC}  a la salida del inversor, se controlará la tensión trifásica que alimenta a la carga.

    
    \item Filtro LCL. En la alimentación de una carga pasiva con un filtro LCL como el de la Fig. \ref{fig.filtroLCL} a la salida del inversor, se controlará la corriente de cada una de las fases que circula por la bobina conectada a la carga \cite{treintaytres, cuarentaycinco}.
    

    \end{enumerate}
    
    \begin{figure}[ht] 
    \begin{center}
    \resizebox{8cm}{!}{\includegraphics{Filtro_L}}
    \end{center}
    \caption{Filtro en L.} 
    \label{fig.filtroL}
    \end{figure}
    
    \begin{figure}[!h]
    \begin{center}
    \resizebox{8cm}{!}{\includegraphics{Filtro_LC}} 
    \end{center}
    \caption{Filtro en LC.} 
    \label{fig.filtroLC}
    \end{figure}

        
    \begin{figure}[!h]
    \begin{center}
    \resizebox{8cm}{!}{\includegraphics{Filtro_LCL}}
    \end{center}
    \caption{Filtro en LCL.}
    \label{fig.filtroLCL}
    \end{figure}
    
    Los prototipos serán de la forma que puede observarse en la Fig. \ref{fig.inversortrifasicoesquematicofiltroRL}. La resistencia serie mostrada en cada fase representa la resistencia del cable de la bobina. 

    \begin{figure}[!h] 
        \centering
        \resizebox{12cm}{!}{\includegraphics{inversortrifasicoesquematicofiltroRL}}
        \caption{Esquema de inversor trifásico con carga RL.}      \label{fig.inversortrifasicoesquematicofiltroRL}
        \end{figure}
    



% Pues eso … los distintos simuladores. Cada uno con mayor nivel de detalle que el anterior.
\clearpage



\clearpage
%%%%%%%%%%%%%%%%%%%%% LOS SEMICONDUCTORES %%%%%%%%%%%%%%%%%%

\chapter{Control del dispositivo} \label{sec.simulaciondeldispositivo}
%%%%%%%%%%%%%%%%%%%%%%%%%%%%%%%%%%%%%%%%%%%%%%%%%%%%%%%%%%%%%%%%%%%%%%%%%%%%%%%%%%%%%%%%%%%%%
%%%%%%%%%%%%%%%%%%%%%%%%%%%%%%%%%%%%%%%%%% DESARROLLO DEL PROYECTO  %%%%%%%%%%%%%%%%%%%%%%%%%%%%%%%%%%%%%%%
%%%%%%%%%%%%%%%%%%%%%%%%%%%%%%%%%%%%%%%%%%%%%%%%%%%%%%%%%%%%%%%%%%%%%%%%%%%%%%%%%%%%%%%%%%%%%
\section{Ecuaciones del sistema} \label{sec.ecuacionesdelsistema}


Sea el sistema de tensiones y corrientes de la Figura \ref{fig.inversorecuaciones}. 

\begin{figure}[!h]
    \begin{center}
        \resizebox{12cm}{!}{\includegraphics{inversorecuaciones}}
        \end{center}
        \caption{Esquema vectorial del circuito. Va0, Vb0 y Vc0 son las tensiones de salida del inversor. Va, Vb y Vc las tensiones que determina el control e ia, ib e ic las variables medidas.}
        \label{fig.inversorecuaciones}
\end{figure}
El control que se diseñará, calculará la tensión que ha de aplicarse a la carga, en la Fig. \ref{fig.inversorecuaciones}, las tensiones $V_{a}$, $V_{b}$ y $V_{c}$. En (\ref{eq.a0b0c0}) se muestra el sistema de ecuaciones de las fases del sistema. En ellas, $V_{a0}$, $V_{b0}$ y $V_{c0}$ son las tensiones de salida del VSC.

\begin{eqnarray}
    \left\{\begin{array}{l}
    V_{a0} = V_{a} + V_{N0} \rightarrow V_{a} = V_{a0} - V_{N0} \\
    V_{b0} = V_{b} + V_{N0} \rightarrow V_{b} = V_{b0} - V_{N0} \\
    V_{c0} = V_{c} + V_{N0} \rightarrow V_{c} = V_{c0} - V_{N0}
\end{array} \right. \label{eq.a0b0c0}
\end{eqnarray}

Sumando las ecuaciones anteriores, resulta:

\begin{equation}
    (V_{a}+V_{b}+V_{c}) =(V_{a0}+V_{b0}+V_{c0}) - 3 \cdot V_{N0} \label{eq.sumaeqabc}
\end{equation}
Se supondrá que los valores de las inductancias y resistencia del cable de las tres bobinas serán iguales, de tal manera que el sistema sea equilibrado:
\begin{equation}
    \left\{\begin{array}{l}
    L_{a} = L_{b} = L_{c} = L \\
    R_{a} = R_{b} = R_{c} = R
    \end{array} \right.
\end{equation}

Así mismo, la tensión que ha de caer en cada una de las bobinas y su resistencia, siendo la impedancia la suma de la inductancia y la resistencia del cable:

\begin{equation}
    \left\{\begin{array}{l}
    V_{a} =  L \cdot \frac{\mathrm{d} i_{a}}{\mathrm{d} t} + R \cdot i_{a}\\
    \null \\
    V_{b} =  L \cdot \frac{\mathrm{d} i_{b}}{\mathrm{d} t} + R \cdot i_{b}\\
    \null \\
    V_{c} =  L \cdot \frac{\mathrm{d} i_{c}}{\mathrm{d} t} + R \cdot i_{c} 
    \end{array} \right. 
    \label{eq.vabc}
\end{equation}



Por lo tanto, suponiendo un sistema trifásico equilibrado, se aplica la ley de Kirchhoff de que la suma de corrientes en un nudo debe ser igual a cero \cite{cuarenta}:
\begin{eqnarray}
    V_{a}+V_{b}+V_{c} = 0 \label{eq.sumadecorrientesiguala0}
\end{eqnarray}

Y operando(\ref{eq.sumaeqabc}) en (\ref{eq.sumadecorrientesiguala0}), resulta:
\begin{eqnarray}
    V_{N0} = \frac{1}{3} \cdot (V_{a0}+V_{b0}+V_{c0}) \label{eq.vn0}
\end{eqnarray}

Las formas de onda de tensión de continua y del armónico fundamental de Va0, Vb0 y Vc0 serán \footnote{Obtenido de los apuntes de la asignatura de Aplicaciones de Electrónica de Potencia. Último acceso: 18/12/2018. Accesible a los alumnos matriculados de la asignatura en la Universidad Pontificia de Comillas.} de la forma en (\ref{eq.formastenionesvi0}), con \textit{i = las tres fases} y $m_a$ el índice de modulación de amplitudes, que variará en función de la tensión que se aplica a la carga,

\begin{equation}
    \left\{\begin{array}{l}
        V_{a0} = m_a \frac{V_{dc}}{2} + m_a \cdot \frac{V_{dc}}{2} \cdot sin(\omega t) \label{eq.formastenionesvi0} \\
    \null \\
    V_{b0} = m_a \frac{V_{dc}}{2} + m_a \cdot \frac{V_{dc}}{2} \cdot sin(\omega t + 120º) \\
    \null \\
    V_{c0} = m_a \frac{V_{dc}}{2} + m_a \cdot \frac{V_{dc}}{2} \cdot sin(\omega t - 120º) \\
    \end{array} \right.
\end{equation}
\newline
Y sustituyendo y operando (\ref{eq.formastenionesvi0}) en (\ref{eq.vn0}):
\begin{eqnarray} 
    V_{N0} = \frac{1}{3} \cdot (3\cdot\frac{m_a \cdot V_{dc}}{2}) = \frac{m_a \cdot V_{dc}}{2} 
\end{eqnarray}


\section{Variables trifásicas y sistemas de referencia} \label{sec.controldevariables}

Las variables que desean controlarse, en cualquier caso, son trifásicas y serán de la forma:

\begin{eqnarray}
v_{i} & =  & \hat{V}_{i} \sin (\omega t + \varphi_{i}) \label{eq.generalsen} 
\end{eqnarray}
Siendo:
\begin{enumerate}
\item $v_{i}$ la tensión de la fase $i$ en un instante de tiempo dado.
\item $\hat{V}_{i}$ la tensión de pico de la fase $i$.
\item $\omega $ la pulsación de la senoidal de salida.
\item $\varphi_{i}$ el adelanto o retraso inicial de la fase $i$.
\end{enumerate}

El control de variables de la forma anterior resulta tedioso por contener funciones senoidales. Además, el ser magnitudes trifásicas obliga a controlar cada rama por separado. La solución a tan ardua tarea es la de aprovechar aparatos matemáticos como son la transformada de Clarke y la transformada de Park. 

%%%%%%%%%%%%%%%%%%%%% TRANSFORMADA DE CLARK %%%%%%%%%%%%%%%%%%

\subsection{Transformada de Clarke} \label{sec.clarke}
La transformada de Clarke consiste en la proyección de tres ejes coplanares, no necesariamente ortogonales entre sí, de componentes $a$, $b$ y $c$, sobre un sistema cartesiano de ejes ortonormales, de componentes $\alpha$, $\beta$ y $0$ \cite{cincuentaytres}, siendo generalmente $0$ la componente homopolar de la transformación. En caso de tener un sistema trifásico en la que los vectores están desfasados entre sí $120$º, se tiene el sistema representado en la Fig. \ref{fig.transformadadeclark}. Si el sistema no tiene un cuarto hilo que conecte el neutro de la carga a tierra, la componente homopolar de la corriente, es nula, lo cuál sucede en casi todos los casos y en el prototipo así será. De ahora en adelante se ignorará la componente homopolar.

\begin{figure}[!h]
\begin{center}
\resizebox{8cm}{!}{\includegraphics{transformadadeclarke}}
\end{center}
\caption{Transformada de Clarke.} \label{fig.transformadadeclark}
\end{figure}

\begin{eqnarray}
U_{\alpha,\beta,0} = C U_{a,b,c} \label{eq.cambiobaseclarke}
\end{eqnarray}


La matriz de cambio de base que resulta de la Fig. \ref{fig.transformadadeclark}, ignorando de nuevo la componente homopolar, es
\begin{eqnarray}
C_{a,b,c\to\alpha,\beta}=
\left( \begin{array}{cccc}
\cos(0) & \cos(2\pi/3)  & \cos(-2\pi/3)\\
\sin(0) & \sin(2\pi/3)  & \sin(-2\pi/3)
\end{array} \right)  \label{eq.c_cos}
\end{eqnarray} 

\begin{eqnarray}
 C_{a,b,c\to\alpha,\beta} =
\left( \begin{array}{cccc}
1 & -1/2  & -1/2\\
0 & \sqrt(3)/2  & -\sqrt(3)/2 \label{eq.Cabcalfabeta}
\end{array} \right)  
\end{eqnarray}

Y a la inversa

\begin{eqnarray}
C^{t}_{a,b,c\to\alpha,\beta} = C_{\alpha,\beta \to a,b,c}=
\left( \begin{array}{cccc}
1 & 0\\
-1/2 & \sqrt(3)/2\\
-1/2 & -\sqrt(3)/2 \label{eq.Calfabetaabc}
\end{array} \right) 
\end{eqnarray}
%%%%%%%%%%%%%%%%%%%%% TRANSFORMADA DE PARK %%%%%%%%%%%%%%%%%%

\subsection{Transformada de Park} \label{sec.park}

La transformada de Park \cite{cincuentaysiete} consiste en aplicar a la transformación de Clarke, a los ejes ortonormales $\alpha,\beta,$ una segunda transformación en forma de rotación alrededor del centro de los ejes coordenados, llevandolos a unos nuevos ejes $d,q$ \cite{cincuentaytres}. Dicha rotación puede ser fija, un ángulo $\theta$, o bien, si el giro es constante, los ejes pueden rotar a velocidad angular $\omega$. La transformación de Park que traslada los ejes ortonormales $\alpha,\beta$ a ejes $dq$ puede expresarse de las formas
La representación gráfica puede observarse en la Fig. \ref{fig.transformadadepark}.

\begin{figure}[!h]
\begin{center}
\resizebox{8cm}{!}{\includegraphics{transformadadeparke}}
\end{center}
\caption{Transformada de Park.} \label{fig.transformadadepark}
\end{figure}

\begin{enumerate}
\item Si la rotación de los ejes se produce un ángulo $\theta$:

\begin{eqnarray}
\left(\begin{array}{cc}
u_{d}\\
u_{q}
\end{array}\right) =
\left( \begin{array}{cccc}
\cos(\theta) & \sin(\theta)\\
-\sin(\theta) & \cos(\theta)
\end{array} \right)
\left(\begin{array}{cc}
u_{\alpha}\\
u_{\beta}
\end{array}\right)  \label{eq.Pdqalfabeta_teta}
\end{eqnarray}

\item Si la rotación de los ejes se produce a velocidad constante $\omega$:

\begin{eqnarray}
\left(\begin{array}{cc}
u_{d}\\
u_{q}
\end{array}\right) =
\left( \begin{array}{cccc}
\cos(\omega t) & \sin(\omega t)\\
-\sin(\omega t) & \cos(\omega t)
\end{array} \right)
\left(\begin{array}{cc}
u_{\alpha}\\
u_{\beta}
\end{array}\right)   \label{eq.Pdqalfabeta_omega}
\end{eqnarray}

\end{enumerate}

Y la matriz de paso que relaciona los ejes iniciales coplanarios $A$,$B$ y $C$ con los nuevos ejes $d,q$ es:

\begin{eqnarray}
 P_{a,b,c\to d,q}=
\left( \begin{array}{cccc}
\cos(\omega t) & \sin(\omega t)\\
-\sin(\omega t) & \cos(\omega t)
\end{array} \right)
C_{a,b,c \to \alpha, \beta}\label{eq.PabcdqC}
\end{eqnarray}  


Y sustituyendo  (\ref{eq.Cabcalfabeta}) en (\ref{eq.PabcdqC})

\begin{eqnarray}
 P_{a,b,c\to d,q}=
\left( \begin{array}{cccc}
(\frac{2}{3})\cos(\omega t) & (\frac{2}{3})\cos(\omega t - 2\pi/3) & (\frac{2}{3})\cos(\omega t + 2\pi/3)\\
-(\frac{2}{3})\sin(\omega t) & -(\frac{2}{3})\sin(\omega t - 2\pi/3) & -(\frac{2}{3})\sin(\omega t + 2\pi/3)\end{array} \right)\label{eq.Pabcdq}
\end{eqnarray}  

Dado que los vectores en $A$,$B$ y $C$ coplanarios pueden no tener la misma magnitud que en los ejes $d-q$, se aplica una homotecia a (\ref{eq.Pabcdq}), incluyendo también la componente homopolar:
\begin{eqnarray}
    P_{a,b,c\to d,q,0}= K
   \left( \begin{array}{cccc}
   (\frac{2}{3})\cos(\omega t) & (\frac{2}{3})\cos(\omega t - 2\pi/3) & (\frac{2}{3})\cos(\omega t + 2\pi/3)\\
   -(\frac{2}{3})\sin(\omega t) & -(\frac{2}{3})\sin(\omega t - 2\pi/3) & -(\frac{2}{3})\sin(\omega t + 2\pi/3)\\
    a & a & a \end{array} \right)
   \label{eq.KPabcdq}
   \end{eqnarray}  

Con:
   \begin{equation}
    P_{d,q \to a,b,c}  = P_{a,b,c\to d,q,0}^{-1} = \frac{(Adj(P_{a,b,c\to d,q}))^{t}}{ \left\lvert P_{a,b,c\to d,q }\right\rvert } \label{eq.Pdq0abc}
\end{equation}

Más representativamente, transformando un vector $(x_a, x_b, x_c)$ en uno $(x_d,x_q,x_0)$ sería:

\begin{eqnarray}
    \left(\begin{array}{cc}
    x_{d}\\
    x_{q}\\
    x_{0}
    \end{array}\right) = K
    \left( \begin{array}{cccc}
    (\frac{2}{3})\cos(\omega t) & (\frac{2}{3})\cos(\omega t - 2\pi/3) & (\frac{2}{3})\cos(\omega t + 2\pi/3)\\
    -(\frac{2}{3})\sin(\omega t) & -(\frac{2}{3})\sin(\omega t - 2\pi/3) & -(\frac{2}{3})\sin(\omega t + 2\pi/3)\\
     a & a & a \end{array} \right)
       \left(\begin{array}{cc}
        x_{a}\\
        x_{b}\\
        x_{c}
        \end{array}\right)\label{eq.KPabcdq2}
    \end{eqnarray}
    

De esta manera, pueden distinguirse distintos casos en funcion del valor de K en (\ref{eq.KPabcdq}).


\begin{itemize}
\item $K = 1$ y $a = 1/3$ : Se denomina invariante en módulo. El módulo del número complejo $u_d + ju_q$ es igual a la amplitud de las variables trifásicas.
\item $K = 1/\sqrt(2)$ y $a = 1/3$  : Se denomina invariante en valor eficaz. El módulo del número complejo $u_d + ju_q$ es igual al valor eficaz de las variables trifásicas. 
\item $K = \sqrt{3/2}$ y y $a = \sqrt(2)/3$ : Se denomina invariante en potencia. Modo de funcionamiento del prototipo. En este caso, respecto de la potencia instantanea se cumple \newline $p(t) = v_a(t)\cdot i_a(t) + v_b(t)\cdot i_b(t) + v_c(t)\cdot i_c(t) = v_d(t)\cdot i_d(t) + v_q(t)\cdot i_q(t) + v_0(t)\cdot i_0(t)$
\end{itemize}

De esta manera, teniendo las matrices anteriores de cambio de variable, se pueden escribir modelos trifásicos de manera sencilla en solo dos coordenadas cartesianas (omitiendo la secuencia homopolar). Se puede convertir así, un vector que gira a velocidad constante alrededor de un punto fijo en un sistema de coordenadas estático, en un vector fijo en otro sistema de referencia que rote a dicha velocidad constante entorno al mismo punto fijo.


El prototipo que se diseña y fabrica en el presente proyecto es el que consiste en un VSC conectado a una bobina trifasica. La variable a controlar será la intensidad por las fases, que llevados a ejes $d - q$, consiste en el control de las variables $i_{d}$ e $i_{q}$, realizando así un diseño de control de corriente\cite{treintaydos}.

Con la transformada de Park invariante en potencia, las potencias activas y reacticas instantaneas son \cite{cuarentaydos}:

\begin{equation}
    p = v_{d} \cdot i_{d} + v_{q} \cdot i_{q} \label{eq.p}  
\end{equation}
\begin{equation}
    q = v_{d} \cdot i_{q} - v_{q} \cdot i_{d}\label{eq.q}  
\end{equation}


\section{Modelo en ejes d-q de una bobina trifásica} \label{sec.modeloenejesdqdeunabobinatrifasica}

Debido a las transformaciones de magnitudes trifásicas, alternas, variables en el tiempo, en magnitudes constantes, es necesario analizar el comportamiento de los componentes en los nuevos ejes tras la transformación \cite{treintaytres,treintayocho,treintaynueve}.
En particular, las bobinas en los nuevos ejes $d$ y $q$, tendrán un modelo peculiar, en el que se relacionan las corrientes y tensiones de los nuevos ejes entre sí, Fig. \ref{fig.control-corriente-acoplamiento}.

\begin{figure}[!h]
    \begin{center}
    \resizebox{9cm}{!}{\includegraphics{control-corriente-acoplamiento}}
    \end{center}
    \caption{Diagrama de control de corriente en ejes d-q \cite{doce}}
    \label{fig.control-corriente-acoplamiento}
    \end{figure}

    
Sea (\ref{eq.desacopladoejeabc}) la ecuación que rige el comportamiento de las tensiones de las bobinas en los ejes $abc$. Siendo $v_a, v_b$ y $v_c$ las tensiones a las que están sometidas las bobinas en cada fase.   

\begin{equation}
\left[\begin{array}{c}u_{a} \\ u_{b} \\ u_{c}\end{array}\right] = R \cdot \left[\begin{array}{c}i_{a} \\ i_{b} \\ i_{c}\end{array}\right]+ L \cdot \frac{d}{dt} \left[\begin{array}{c}i_{a} \\ i_{b} \\ i_{c}\end{array}\right]  \label{eq.desacopladoejeabc}
\end{equation}
Aplicando la matriz de transformación en (\ref{eq.Pdq0abc}) resulta (\ref{eq.desacopladoejedq0}) (para simplificar $P_{d,q\to a,b,c} = P$)

\begin{equation}
P \cdot \left[\begin{array}{c}v_{d} \\ v_{q} \\ v_{0}\end{array}\right] = R \cdot P \cdot \left[\begin{array}{c}i_{d} \\ i_{q} \\ i_{0}\end{array}\right]+ L \cdot P \cdot \frac{d}{dt} \left[\begin{array}{c}i_{d} \\ i_{q} \\ i_{0}\end{array}\right]  \label{eq.desacopladoejedq0}
\end{equation}


y despejando las tensiones matricialmente en (\ref{eq.desacopladoejedq0}):
\begin{equation}
\underbrace{P^{-1} \cdot P}_{\mathbf{I}_{33}} \cdot \left[\begin{array}{c}v_{d} \\ v_{q} \\ v_{0}\end{array}\right] = P^{-1} \cdot R \cdot P \cdot \left[\begin{array}{c}i_{d} \\ i_{q} \\ i_{0}\end{array}\right]+ P^{-1} \cdot L \cdot P \cdot \frac{d}{dt} \left[\begin{array}{c}i_{d} \\ i_{q} \\ i_{0}\end{array}\right]  \label{eq.desacopladoejedq02}
\end{equation}
Operando las operaciones matriciales y despejando las tensiones en (\ref{eq.desacopladoejedq02}):

\begin{equation}
    \left\{\begin{array}{l}
    v_{d} = R \cdot i_{d} + \frac{\mathrm{d}({L \cdot i_{d}})}{\mathrm{d} t} - \omega_{1} \cdot L \cdot i_{q}
    \\
    \null \\
    v_{q} = R \cdot i_{q} + \frac{\mathrm{d}({L \cdot i_{q}})}{\mathrm{d} t} + \omega_{1} \cdot L\cdot i_{d} \end{array} \right.\label{eq.vdvq} 
\end{equation}


Pueden escribirse las ecuaciones de (\ref{eq.vdvq}) en valores por unidad, definiendo:

\begin{equation}
    \left\{\begin{array}{l} 
    v_{pu}= \frac{v}{U_{b}}, \;\; i_{pu}= \frac{i}{I_{b}}, \;\; S_{b} = 3 U_{b}I_{b}  \\ \null  \\
    L_{pu} = \frac{L}{L_{b}}, \;\; R_{pu} = \frac{R}{Z_{b}},\;\; Z_{b} = 3\frac{U_{b}^{2}}{S_{b}} \\ \null \\
    Z_{b} = L_{b} \omega_{b}, \;\; \omega_{1,pu} = \frac{\omega_{1}}{\omega_{b}}
    \end{array} \right.
    \end{equation}

De manera que se obtienen las ecuaciones de tensión de la inductancia y resistencia serie en valores por unidad siguientes:
\begin{equation}
    \left\{\begin{array}{l}
    \omega_b \cdot v_{d,pu} = \omega_b \cdot (R_{pu} \cdot i_{d,pu}) + \frac{\mathrm{d}({L_{pu} \cdot i_{d,pu}})}{\mathrm{d} t} - \omega_b (\omega_{1} \cdot L_{pu} \cdot i_{q,pu})
    \\
    \null \\
    \omega_b \cdot v_{q,pu} = \omega_b \cdot (R_{pu} \cdot i_{q,pu}) + \frac{\mathrm{d}({L_{pu} \cdot i_{q,pu}})}{\mathrm{d} t} + \omega_b (\omega_{1} \cdot L_{pu} \cdot i_{d,pu})
    \end{array} \right. \label{eq.vdvqpu} 
\end{equation}


Observando el comportamiento de (\ref{eq.vdvq}), se comprueba que en el régimen transitorio, las magnitudes de ejes $d$ y $q$ se ven influenciadas entre sí. Sin embargo, en régimen permanente las componentes de eje $d$ solo dependen de las componentes de eje $q$ y que las componentes de eje $q$ solo dependen de las componentes de eje $d$ \cite{cuarentaytres}.

\begin{equation}
    v_{d} = - \omega_{1} \cdot L \cdot i_{q} \label{eq.vdsimplificadaregimenpermanente} 
\end{equation}

\begin{equation}
    v_{q} =\omega_{1} \cdot  L \cdot i_{d} \label{eq.vqsimplificadaregimenpermanente} 
\end{equation}


\section{Magnitudes Base} \label{sec.magnitudesbase}

Es necesario determinar una serie de magnitudes eléctricas base, dado que más adelante, se diseñará un regulador PID tanto en tiempo continuo como en tiempo discreto. Haciendo uso de la documentación técnica de los componentes \cite{diecisiete}, sabemos que la tensión nominal del inversor, en su entrada de tensión continua, es de $V_{dc} = 48V$. La corriente máxima del VSC es de $I_{máx}^{inversor} = 10A$. Las bobinas admiten una corriente máxima de $I_{máx}^{L} = 5A$. La inductancia de las bobinas (por fase) es de $L = 5.881mH$. La resistencia de la inductancia (por fase) es de $R_{L} = 1.1$ $\Omega$.

Así mismo, dado que consiste en un prototipo para docencia, se trabajará con una intensidad por fase máxima de 2.5A. También, haciendo uso de (\ref{eq.tensioncompuestaindicedemodulacion}), sustituyendo en está, por seguridad, un índice de modulación de amplitudes máximo de $m_a=1$, tendremos una magnitud de tensión base para la tensión compuesta, es decir, tensión entre dos fases,

\begin{equation}
V_{f-f} = 0.61 \cdot m_a \cdot Vdc = 0.61 \cdot 1 \cdot 48 = 29.28V \label{eq.tensioncompuestaindicedemodulacion}
\end{equation}
Y la tensión que soporta una sola bobina respecto al neutro de la fuente de tensión, será:
\begin{equation*}
    V_{f} = \frac{V_{f-f}}{\sqrt{3}}
\end{equation*}

Las magnitudes base serán:

\begin{eqnarray*}
    V_{dc} &=& 48V \\
    V_{b} &=& V_{f-f} = 29.28V \\
    I_{b} &=&2.5 A\\
    S_{b} &=& \sqrt(3) \cdot V_{b} \cdot I_{b}\\
    Z_{b} &=& \frac{V_{b}}{I_{b}}\\
    \omega_{b} &=& 2 \cdot \pi \cdot f =  2 \cdot \pi \cdot 50\\
    L_{b} &=& \frac{Z_{b}}{\omega_{b}}
\end{eqnarray*}



\chapter{Diseño, simulación e implantación de controles P - PI} \label{sec.controlespid}

Los controles sobre variables se han venido usando durante siglos para poder controlar todo tipo de variables. Los primeros controles se realizaban de manera analógica, mediante mecanismos \cite{cincuentaycuatro}.
Hoy en día, se implementan los controles, en su inmensa mayoria, digitalmente. La implantación de un control de manera digital requiere de un computador (o microcomputador) para realizar todas las operaciones pertinentes. (Véase la Sección \ref{sec.microcontroladores}). 
Existen numerosos tipos de controles de variables, como por ejemplo, controles proporcionales, integrales y diferenciales (PID), o controles en espacio de estado, mediante realimentación de estado. 

En el presente proyecto se realizará un control PI sobre la corriente de las bobinas, no obstante, es posible realizar controles PID completos \cite{treintaydos}, a pesar de que estos sistemas pueden resultar muy ruidosos por la conmutación de los semiconductores (perjudicial utilizando un control diferencial). El diseño de un control PI en un sistema de ejes rotacionales $d-q$ proporcionan un error nulo en regimen permanente en seguimiento de referencia \cite{treintaytres}

La función de transferencia entre referencia ($R$) y salida ($Y$) de todo sistema se define como en (\ref{eq.ftrefsalida}) en tiempo continuo utilizando la variable de Laplace ($s$).

\begin{equation}
    F(s) = \frac{R(s)}{Y(s)} \label{eq.ftrefsalida}
\end{equation}

Sea el diagrama de bloques de un sistema en tiempo continuo representado en la Fig. \ref{fig.DiagramadebloquesControlPID}, en la cuál, el control diseñado es $C(s)$, la planta del sistema es $P(s)$ y la realimentación es $H(s)$. Así mismo:
\begin{itemize}
        \item $r(t)$ : es la señal de referencia del control.
        \item $e(t)$ : es el error entre la medida de la salida ($y_{m}(t)$) y la salida ($y(t)$).
        \item $u(t)$ : es el mando que se aplica a la planta.
        \item $y(t)$ : es la salida del sistema.
        \item $y_{m}(t)$ : es la medida de la salida.
\end{itemize}

Se denomina $G(s)$ a la función de transferencia entre referencia y salida en lazo abierto, véase (\ref{eq.ecuacionG}) . Mientras que se denomina $F(s)$, a la función de transferencia entre referencia y salida en lazo cerrado, veáse (\ref{eq.ecuacionF}).

\begin{figure}[!h]
    \begin{center}
        \resizebox{10cm}{!}{\includegraphics{DiagramadebloquesControlPID}}
        \end{center}
        \caption{Diagrama de bloques de un control.}
        \label{fig.DiagramadebloquesControlPID}
\end{figure}



\begin{equation}
    G(s) = C(s) \cdot P(s) \cdot H(s)\label{eq.ecuacionG}
\end{equation}


\begin{equation}
    F(s) = C(s) \cdot \frac{P(s)}{1+G(s)} \label{eq.ecuacionF}
\end{equation}

El diseño de controles PID se realiza mediante respuesta temporal o respuesta en frecuencia. La respuesta de ambos controles en función de los valores de los parámetros del control diseñado. Entre las principales cuestiones a tener en cuenta serán, respecto a su respuesta temporal para un control PI, analizando su respuesta a un escalón en referencia en lazo cerrado:

\begin{itemize}
\item Error de seguimiento.
\item Sobrepaso.
\end{itemize}

Y respecto su respuesta en frecuencia, analizando el diagrama de Black en lazo abierto:

\begin{itemize}
    \item Margen de ganancia.
    \item Pulsación de oscilación.
\end{itemize}


El diseño por respuesta en frecuencia se relaciona con su respuesta temporal mediante:

\begin{equation*}   
    \text{Amortiguamiento} \Longleftrightarrow \text{Margen de ganancia}
\end{equation*}

\begin{equation*}
    \text{Velocidad} \Longleftrightarrow \text{Pulsación de cruce}
\end{equation*}

\section{Diseño de un control P - PI en tiempo continuo}\label{sec.controlPIDtiempocontinuo}

En primer lugar, se diseñará un control en tiempo continuo con la forma de la Fig. \ref{fig.diagramacontrolPIDcontinuocite}.

\begin{figure}[!h]
    \begin{center}
        \resizebox{10cm}{!}{\includegraphics{diagramacontrolPIDcontinuocite}}
        \end{center}
        \caption{Diagrama de bloque de un control en tiempo continuo \cite{treintayseis}.}
        \label{fig.diagramacontrolPIDcontinuocite}
\end{figure}

El retardo sobre la planta se realiza multiplicando ésta por $e^{-\frac{Ts}{2} \cdot s}$ \cite{treintayseis}, emulando así el retardo del muestreo en tiempo continuo.

El diseño de un control en tiempo continuo requiere de una variable que varie con el tiempo y que lo represente. Se utilizará para el diseño en tiempo continuo del control la variable de Laplace ($s$). 

Su diseño puede realizarse tanto en formato serie como en formato paralelo. Los parametros en formato paralelo son:
\begin{itemize}
    \item $K$ : Parámetro correspondiente a la parte proporcional del control.
    \item $T_{i}$ : Parámetro correspondiente a la parte integral del control.
    \item $b$ : Parámetro de ponderación de la referencia correspondiente a la parte integral del control.
\end{itemize}

El diagrama de bloques del control PI en formato paralelo puede observarse en la Fig. \ref{fig.DiagramadebloquesControlPI}.
\begin{figure}[!h]
    \centering
    \resizebox{12cm}{!}{\includegraphics{DiagramadebloquesControlPI}}
    \caption{Diagrama de bloques de un control PI en formato paralelo.}
    \label{fig.DiagramadebloquesControlPI}
\end{figure} 

La función de transferencia, entre referencia y salida, de un control PI en formato paralelo y tiempo continuo tendrá, como función de transferencia del control en (\ref{eq.funciondetransferenciatiempocontinuocontrolparalelo}), con $b = 1$, lo cuál se utiliza en el control del prototipo:

\begin{equation}
    C_{paralelo}(s) = K\cdot (b + \frac{1}{Ti\cdot s}) \label{eq.funciondetransferenciatiempocontinuocontrolparalelo}
\end{equation}

Los parametros en formato serie son:
\begin{itemize}
    \item $K_{p}$  : Parámetro correspondiente a la parte proporcional del control.
    \item $I$ : Parámetro correspondiente a la parte integral del control.
\end{itemize}  

La función de transferencia del control tendrá la forma de (\ref{eq.funciondetransferenciatiempocontinuocontrolserie});

\begin{equation}
    C_{serie}(s) = K_p \cdot  \frac{1+I\cdot s}{I\cdot s} \label{eq.funciondetransferenciatiempocontinuocontrolserie}
\end{equation}

La ecuación que relaciona ambos formatos es:

\begin{equation*}\label{eq.paraleloaserie}
    \mu = 1
\end{equation*}

\begin{equation*}
K = \mu \cdot  K_{p} 
\end{equation*}

\begin{equation*}
T_{i} = \mu \cdot  I
\end{equation*}

La obtención de las funciones de transferencia en lazo abierto y lazo cerrado entre referencia y salida, se realizarán mediante la ejecución de un fichero de simulink, representado en la Fig. \ref{fig.Ldqp_simulink}, en el cuál se implementará un modelo de acoplamiento y desacoplamiento de las bobinas en ejes $d-q$.  De esta manera, al ejecutar el script en Matlab que compile el fichero de simulink, se obtendrán las funciones de transferencia en dichos ejes, así como la planta en tiempo continuo. No obstante, es necesario determinar unas magnitudes base (véase la Sección \ref{sec.magnitudesbase}) para los componentes, como las bobinas, resistencias, tensiones, intensidades, etc. 

\begin{figure}[!h]
    \centering
    \resizebox{12cm}{!}{\includegraphics{Ldqp_simulink}}
    \caption{Simulink para la obtención de la planta en tiempo continuo.}
    \label{fig.Ldqp_simulink}
\end{figure} 

\clearpage

\subsection{Diseño de un control PI en tiempo discreto} \label{sec.controlPIDtiempodiscretp}

El diseño de un control PI en tiempo discreto puede realizarse de distintas maneras:
\begin{itemize}
    \item Mediante la discretización de un control en tiempo continuo.
    \item Mediante el diseño íntegro en tiempo discreto.
\end{itemize}

\subsection{Discretización de un control en tiempo continuo}

La discretización de un sistema en tiempo continuo a tiempo discreto, requiere de una variable en tiempo discreto y de un periodo de muestreo especificado. Utilizando la variable en tiempo discreto $z$ y un periodo de muestreo $T_{s}$, puede hallarse una ecuación que relacione los polos y ceros de un sistema en tiempo continuo con los correspondientes en tiempo discreto, véase  (\ref{eq.relacioncontinuodiscretopolosyceros}).

\begin{equation}
z = e^{s \cdot T_{s}} \label{eq.relacioncontinuodiscretopolosyceros}
\end{equation}

Así mismo, es posible discretizar el control mediante otros metodos, como son:
\begin{itemize}
    \item Aproximación de la derivada o de la integral: Utilizando una transformación de tiempo continuo a tiempo discreto haciendo uso de las variables $s$ y $z$.
    \item Transformación invariante: Suponiendo que existe un retenedor en la entrada, por ejemplo, de orden 0, de manera que la discretización mantiene un valor constante durante el periodo de muestreo.
\end{itemize}

El diagrama de bloques de un control en tiempo discreto puede observarse en la Fig. \ref{fig.DiagramadebloquesControldiscreto}. Es necesaria la implantación de dos bloques que permiten el paso de tiempo continuo a discreto y viceversa. Dichos bloques son:

\begin{figure}[!h]
    \begin{center}
    \resizebox{8cm}{!}{\includegraphics{DiagramadebloquesControldiscreto}}
    \caption{Diagrama de bloques de un control en tiempo discreto.}
    \label{fig.DiagramadebloquesControldiscreto} 
    \end{center}
\end{figure}

\begin{itemize}
    \item D/A : Es el bloque retenedor. Consiste en mantener un cierto valor durante el tiempo de muestreo, de manera que la salida ``\textit{vea}'' una señal continua.
    \item A/D : Es el bloque muestreador, permite convertir los valores en tiempo continuo de la salida a valores discretos, cada periodo de muestreo.
\end{itemize}

El método de discretización utilizado será el de la derivada en adelanto de la integral, cuya definición se demuestra en  (\ref{eq.regladerivadaenadelantointegral})\footnote{Obtenido de los apuntes de la asignatura de Control Digital. Último acceso: 20/06/2019. Material accesible para los alumnos matriculados en la asignatura de Control Digital en la Universidad Pontificia de Comillas.\label{note1}}, donde $k$ representa el instante actual y $k-1$ el instante un periodo de muestreo ($T_{s}$) anterior.

\begin{equation}
    y[k] = \int_{0}^{k\cdot T_{s}} u(\tau) \,\mathrm{d}\tau \simeq \int_{0}^{(k-1)\cdot T_{s}} u(\tau) \,\mathrm{d}\tau + u\cdot[k-1]\cdot T_{s} \label{eq.regladerivadaenadelantointegral}
\end{equation}
Siendo,
\begin{equation}
    \int_{0}^{(k-1)\cdot T_{s}} u(\tau) \,\mathrm{d}\tau = y \cdot [k-1]
\end{equation}
Y aplicando la transformada Z resulta la igualdad de la ecuación,

\begin{equation}
    \frac{1}{s}=\frac{z^{-1}\cdot T_{s}}{1-z^{-1}} = \frac{T_{s}}{z-1} \label{eq.igualdadsaz}
\end{equation}

De esta manera, el control aplicado, utilizando  (\ref{eq.igualdadsaz}), discretizamos el control en $z$. 

La determinación del periodo de muestreo no es trivial, dado que, como se explica en la Sección \ref{sec.muestreodesenales}, es necesario que el periodo de muestreo sea mayor que el periodo de un ciclo de PWM. Por lo tanto, si deseamos que el inversor conmute a una frecuencia $f_{c}$, es decir, que el periodo de conmutación sea de $T_{c} = \frac{1}{f_{c}}$, el periodo de muestreo ha de ser:
\begin{equation}
    T_{s} = k \cdot T_{c} \label{eq.tcats}
\end{equation}

Para evitar efectos indeseados en el muestreo, trabajando por debajo de la frecuencia de nyquist, evitando así aliasing, calcularemos el periodo de muestreo de  (\ref{eq.tcats}) con $k = 3$, de manera que el ADC inicie la conversión de valores cada tres periodos de conmutación del PWM, por lo tanto tendremos que el periodo de muestreo será,
\begin{equation}
    T_{s} = k \cdot T_{c} = 3 \cdot \frac{1}{f_{c}} = 3\cdot\frac{1}{10kHz} = 0.3 ms    \label{eq.valorts}
\end{equation}

Considerando por lo tanto un periodo de muestreo de $T_{s} = 0.3 ms$ y una pulsación de cruce de $\omega_{o} = 500 rad/s$ (utilizada en las simulaciones, pudiendo ser diferente dicho valor),
\begin{equation*}
    \omega_{o}\cdot T_{s} = 500 rad/s \cdot 0.0003 s = 0.15 rad
\end{equation*}

Lo cuál se considera un periodo de muestreo mediano, al estar comprendido entre [0,1], siguiendo un criterio de clasificación del periodo de muestreo \footref{note1}. Siendo un periodo de muestreo mediano, el retraso que produce la acción integral debe de estar comprendido entre 5º y 30º. El retraso de fase del control integral es de $\phi_{pi} = 15.3428$º para los valores anteriormente fijados, verificando así la validez del control. Al discretizar la acción integral con un periodo de muestreo mediano, el método de discretización utilizado es correcto.

La aplicación de (\ref{eq.igualdadsaz}) a la función de transferencia en tiempo continuo del control PI desarrollado en la Sección \ref{sec.controlPIDtiempocontinuo}, es decir, discretizando el control de (\ref{eq.funciondetransferenciatiempocontinuocontrolserie}) de la página \pageref{eq.funciondetransferenciatiempocontinuocontrolserie}, resulta:

\begin{equation}
    C(z) = Kp \cdot \frac{I + \frac{z-1}{T_{s} }}{I\cdot\frac{z-1}{T_{s}}} = Kp \cdot \frac{I+\frac{T_{s}}{z-1}}{I}
\end{equation}

El bloque de Simulink que permite la implantación software y hardware del control PID se muestra en la Fig. \ref{fig.bloquepid}. En dicha figura se muestra como ha de introducirse el valor de la integral de tal manera que responda al comportamiento de las ecuaciones anteriormente descritas.

\begin{figure}[!h]
    \centering
    \resizebox{10cm}{!}{\includegraphics{bloquepid}}
    \caption{Bloque PID de implantación Software y Hardware.}
    \label{fig.bloquepid}
\end{figure} 

\section{Descripción del simulador} \label{sec.simulador}

El simulador diseñado en Simulink permite configurar:

\begin{itemize}
    \item Salidas por el scope. Tanto en formato de magnitudes teóricas como las que se obtendrían por el osciloscopio al ejecutar el programa en el LaunchPad.
    \item Referencias en ejes d y q.
    \item Lazo abierto, lazo cerrado y acoplamiento
    \item Tipo de control, P o PI.
\end{itemize}
En dicho simulador se han introducido las magnitudes base descritas en la Sección \ref{sec.magnitudesbase}.  
La forma de introducir las opciones anteriormente descritas se realiza mediante una interfaz (idéntica a la que se utiliza para el hardware en el prototipo) como las mostradas en la Fig. \ref{fig.configuraciones_simulink}.

\begin{figure}[!h]
    \centering
    \resizebox{10cm}{!}{\includegraphics{configuraciones_simulink}}
    \caption{Interfaz para la configuración del simulador. Idéntica a la que se utiliza en el hardware del prototipo. El desarrollo se describe en el Anexo II.}
    \label{fig.configuraciones_simulink}
\end{figure} 

La congifuración de los parámetros del Simulink se hará como se muestra en la Fig. \ref{fig.configuraciones_simulador_simulink}.

\begin{figure}[!h]
    \centering
    \resizebox{10cm}{!}{\includegraphics{configuraciones_simulador_simulink}}
    \caption{Configuraciones de los parametros del simulador.}
    \label{fig.configuraciones_simulador_simulink}
\end{figure} 

Cabe destacar que para que la simulación se ejcute correctamente, el valor del \textit{Fixed Step} debe ser necesariamente un valor más pequeño que el periodo más corto que tenga el sistema y debe ser un divisor sub-multiplo suyo. Dado que el periodo más pequeño es el periodo de conmutación, de $1 e-4 s$, el \textit{Fixed Step} que simula correctamente el sistema, con una buena calidad en las salidas, con un tiempo de simulación razonable, de $1e-6s$.

\clearpage
\section{Control P}
\subsection{Diseño de un control P. Tiempo Continuo.} \label{sec.controlp}

Las ecuaciones de diseño de un control P en tiempo continuo mediante respuesta en frecuencia se describen a continuación.
\begin{eqnarray}
    \omega_{0} &=& -180  + \phi_{m} - \angle P\\ \label{eq.pulsacioncrucep}
    K &=& \frac{1}{\left\lvert P (\omega_{0})\right\rvert }\\
\end{eqnarray}

Dado que el control proporcional es una ganancia K:
\begin{eqnarray}
    C(s) = C(z) = K
\end{eqnarray}


Probando distintos valores para el margen de fase de (\ref{eq.pulsacioncrucep}), se obtienen las distintas respuestas temporales y de estabilidad en lazo abierto de las Fig. \ref{fig.resptempPid}, \ref{fig.resptempPiq} y \ref{fig.repestabilidadP}.

\begin{figure}[!h]
    \centering
    \resizebox{13cm}{!}{\includegraphics{resptempPid_2}}
    \caption{Respuesta temporal de eje d de un Control P en tiempo continuo. Pulso de Id = 1 en 0.3s. Pulso de Iq = 1 en 0.8s.}
    \label{fig.resptempPid}
\end{figure} 

\begin{figure}[!h]
    \centering
    \resizebox{13cm}{!}{\includegraphics{resptempPiq_2}}
    \caption{Respuesta temporal de eje q de un Control P en tiempo continuo.Pulso de Id = 1 en 0.3s. Pulso de Iq = 1 en 0.8s.}
    \label{fig.resptempPiq}
\end{figure} 

\begin{figure}[!h]
    \centering
    \resizebox{12cm}{!}{\includegraphics{respestabilidadP_2}}
    \caption{Respuesta en frecuencia de lazo abierto de un Control P en tiempo continuo.}
    \label{fig.repestabilidadP}
\end{figure} 
Los resultados de los márgenes de fase y ganancias se recogen en la Tabla \ref{tab.datoscontrolP}:

\begin{table}[!h]
    \begin{center}
    \begin{tabular}{|c|c|c|} \hline\hline\hline
        \textbf{Margen de Fase} ($\phi_{m}$) & \textbf{Pulsación de Cruce ($\omega_{0}$)} & \textbf{Ganancia del Control P ($K$)} \\ \hline
    15º & 8867 rad/s & 7.7138\\ \hline
    30º & 7155.3 rad/s & 6.2255\\ \hline   
    45º & 5463.8 rad/s & 4.755\\ \hline
    60º & 3816.7 rad/s & 3.3238\\ \hline 

    \end{tabular}
    \end{center}
    \caption{Tabla con los datos de un Control P.} \label{tab.datoscontrolP}
    \end{table}
    

\clearpage
\subsection{Simulador de un Control P. Tiempo Discreto.} \label{sec.simuladorcontrolP}

Aplicando una configuración como la que puede observarse en la Fig. \ref{fig.simulador_p}, se obtiene las salidas que pueden observarse en la Fig. \ref{fig.scope_simulador_p}. La Tabla \ref{tab.datossimuladorp} muestra los datos del control aplicado al simulador.
Puede observarse que existe un error en regimen permanente de seguimiento de referencias en ejes $d-q$ debido a que es un control proporcional, sin integrador.
\begin{figure}[!h]
    \centering
    \resizebox{12cm}{!}{\includegraphics{simulador_p}}
    \caption{Configuración de un simulador con un control P. Idéntica a la que se utiliza en el simulador. El desarrollo se describe en el Anexo II.}
    \label{fig.simulador_p}
\end{figure} 



\begin{table}[!h]
    \begin{center}
    \begin{tabular}{|c|c|} \hline\hline\hline
    \textbf{Variable} & \textbf{Valor} \\ \hline
    Margen de Fase ($\phi_{m}$) & 30º \\ \hline
    Ponderación de la referencia ($b$) & 1 \\ \hline 
    Constante de proporcionalidad ($K$) & 0.5 \\ \hline 
    Índice de modulación ($m$) & 0.5 \\ \hline
        
    \end{tabular}
    \end{center}
    \caption{Tabla con los valores de un Control P aplicado al simulador.} \label{tab.datossimuladorp}
\end{table}
\begin{figure}[!h]
    \centering
    \resizebox{13cm}{!}{\includegraphics{scope_simulador_p}}
    \caption{Scope de un simulador con un control P. El bloque PID de simulink se implanta en tiempo discreto. Los valores del control de la simulación se recogen en la Tabla \ref{tab.datossimuladorp}}
    \label{fig.scope_simulador_p}
\end{figure} 

\clearpage

\subsection{Implantación de un Control P en el prototipo} \label{sec.implantacioncontrolP}

La implantación de un Control P implica necesariamente la existencia en regimen permanente de un error de seguimiento de referencia. Se tomaron experimentalmente datos de los errores de seguimiento de eje d, con distintos valores de referencia de eje d y referencia de eje q nula. Las condiciones de funcionamiento del prototipo se registran en la Tabla \ref{tab.datosimplantacioncontrolP}. En todos ellos, se ha utilizado un margen de fase $\phi_{m}$ = 30º. El desacoplo explicado en la Sección \ref{sec.modeloenejesdqdeunabobinatrifasica} de la página \pageref{sec.modeloenejesdqdeunabobinatrifasica}, hace que la corriente de eje q no varíe con la referencia de eje d, de esta manera, la salida de Iq siempre será 0 en regimen permanente cuando la referencia de Iq sea 0. No obstante, dado que el eje d representa una perturbación en el eje q y viceversa, y aplicando un control P, se produce un error de seguimiento y por tanto la salida de Iq es distinta de 0 en régimen permanente. Así mismo, la tensión de alimentación del VSC fue de 48V.

\begin{table}[!h]
    \begin{center}
    \begin{tabular}{|c|c|c|} \hline\hline\hline
    \textbf{Id Referencia (pu)} & \textbf{Id Salida (pu)} & \textbf{Error Id(pu)} \\ \hline
    0.02 & 0.016 & 0.004 \\ \hline
    0.08&  0.03733 & 0.0427\\ \hline   
    0.1 & 0.044 & 0.056\\ \hline
    0.12 & 0.0533 & 0.0667\\ \hline 
    0.14 & 0.06 & 0.08\\ \hline 
    0.16 & 0.0667 & 0.0933\\ \hline 

    \end{tabular}
    \end{center}
    \caption{Tabla con los datos de un Control P. Implantación hardware.} \label{tab.datosimplantacioncontrolP}
\end{table}

En la Fig. \ref{fig.excelcontrolP} se muestran los datos de la Tabla \ref{tab.datosimplantacioncontrolP} en una gráfica.

\begin{figure}[!h]
    \centering
    \resizebox{13cm}{!}{\includegraphics{excelcontrolP}}
    \caption{Gráfica de error de seguimiento de referencia de corriente de eje d con un Control P. Implantación hardware. Los datos del eje y están escalados a un 0.1 del valor en pu.}
    \label{fig.excelcontrolP}
\end{figure} 
\clearpage

\section{Control PI}

\subsection{Diseño de un control PI} \label{sec.controlpi}

Se diseñará el control mediante su respuesta en frecuencia, utilizando la planta en tiempo continuo obtenida en la Sección \ref{sec.controlPIDtiempocontinuo}. Las ecuaciones de diseño por respuesta en frecuencia para un control PI serán:

\begin{eqnarray}
    \phi_{c} & = & -180 + \phi_{m} - \angle P(\omega_{0})\\ \label{eq.pulsacioncrucepi}
    Ac & = & \frac{1}{\left\lvert P (\omega_{0})\right\rvert }\\
    I & = & \tan (\frac{90+\phi_{c}}{\omega_0})\\
    K & = & Ac \cdot \frac{I \cdot \omega_{0}}{\sqrt{1+(I\cdot \omega_{0})^{2}}}\\
\end{eqnarray}

El diseño del control proporcional integral, es decir, de las variables de la función de transferencia del control en  (\ref{eq.funciondetransferenciatiempocontinuocontrolserie2}) se diseñará ajustando los parámetros de pulsación de cruce y de margen de fase. El procecimiento a seguir será el de iteración para ajustar los valores de pulsación de cruce y margen de fase a los valores óptimos.
\begin{equation}
    C_{serie}(s) = K_p \cdot  \frac{1+I\cdot s}{I\cdot s} \label{eq.funciondetransferenciatiempocontinuocontrolserie2}
\end{equation}

Dado que existen dos valores que pueden variar, en una primera instancia se variará el valor del margen de fase. Las Fig. \ref{fig.respuestatemporalfaseconstd}, \ref{fig.respuestatemporalfaseconstq}, \ref{fig.respuestatemporalpulsacionconstd} y \ref{fig.respuestatemporalpulsacionconstq}, muestran las diferentes respuestas temporales en tiempo continuo que podemos tener tomando, por ejemplo, cuatro valores diferentes para el margen de fase (manteniendo constante la pulsación de cruce) y otros cuatro valores diferentes para la pulsación de cruce (mantiendo constante el margen de fase).


\begin{figure}[!h]
    \centering
    \resizebox{12cm}{!}{\includegraphics{respuestatemporalfaseconstd_2}}
    \caption{Respuesta temporal de Id con margen de fase constante con el control PI en tiempo continuo. Pulso de Id = 1 en 0.3s. Pulso de Iq = 1 en 0.8s.}
    \label{fig.respuestatemporalfaseconstd}
\end{figure} 

\begin{figure}[!h]
    \centering
    \resizebox{12cm}{!}{\includegraphics{respuestatemporalfaseconstq_2}}
    \caption{Respuesta temporal de Iq con margen de fase constant con el control PI en tiempo continuo. Pulso de Id = 1 en 0.3s. Pulso de Iq = 1 en 0.8s.}
    \label{fig.respuestatemporalfaseconstq}
\end{figure} 


\begin{figure}[!h]
    \centering
    \resizebox{12cm}{!}{\includegraphics{respuestatemporalpulsacionconstd_2}}
    \caption{Respuesta temporal de Id con pulsación de cruce constante con el control PI en tiempo continuo. Pulso de Id = 1 en 0.3s. Pulso de Iq = 1 en 0.8s.}
    \label{fig.respuestatemporalpulsacionconstd}
\end{figure} 

\begin{figure}[!h]
    \centering
    \resizebox{12cm}{!}{\includegraphics{respuestatemporalpulsacionconstq_2}}
    \caption{Respuesta temporal de Iq con pulsación de cruce constante con el control PI en tiempo continuo. Pulso de Id = 1 en 0.3s. Pulso de Iq = 1 en 0.8s.}
    \label{fig.respuestatemporalpulsacionconstq}
\end{figure} 

Así mismo, las Fig. \ref{fig.respuestablackfaseconstG} y \ref{fig.respuestablackpulsacionconstG} son diagramas de Black que permiten analizar la estabilidad del sistema en lazo abierto.

\begin{figure}[!h]
    \centering
    \resizebox{13cm}{!}{\includegraphics{respuestablackfaseconstG_2}}
    \caption{Respuesta en lazo abierto con margen de fase constante con el control PI en tiempo continuo.}
    \label{fig.respuestablackfaseconstG}
\end{figure} 

\begin{figure}[!h]
    \centering
    \resizebox{13cm}{!}{\includegraphics{respuestablackpulsacionconstG_2}}
    \caption{Respuesta en lazo abierto con pulsación de cruce constante con el control PI en tiempo continuo.}
    \label{fig.respuestablackpulsacionconstG}
\end{figure} 

\clearpage
\subsection{Simulador de un Control PI} \label{sec.simuladorcontrolPI}

Para la simulación del control PI, se aplicará una configuración como la de la Fig. \ref{fig.simulador_pi} a la interfaz creada para el simulador.

\begin{figure}[!h]
    \centering
    \resizebox{12cm}{!}{\includegraphics{simulador_pi}}
    \caption{Configuración aplicada al simulador con un Control PI. Idéntica a la que se utiliza en el simulador. El desarrollo se describe en el Anexo II.}
    \label{fig.simulador_pi}
\end{figure} 


La salida del simulador, aplicando la configuración de la Fig. \ref{fig.simulador_pi} es la que se observa en la Fig. \ref{fig.scope_simulador_pi}. Se puede observar en los saltos que se producen en los ejes cuando su referencia es nula, que las variaciones en eje $q$ son perturbaciones en eje $d$ y viceversa.  Los valores del control PI aplicado al simulador son los descritos en la Tabla \ref{tab.datossimuladorpi}:

\begin{table}[!h]
    \begin{center}
    \begin{tabular}{|c|c|} \hline\hline\hline
    \textbf{Variable} & \textbf{Valor} \\ \hline
    Margen de Fase ($\phi_{m}$) & 50 \\ \hline
    Pulsación de cruce ($\omega_{0}$) & 500 \\ \hline
    Ponderación de la referencia ($b$) & 1 \\ \hline 
    Constante de proporcionalidad ($K$) & 0.2582 \\ \hline 
    Constante de integración ($I$) & 0.0013 \\ \hline 
    Índice de modulación ($m$) & 0.5 \\ \hline

    \end{tabular}
    \end{center}
    \caption{Tabla con los valores de un Control PI aplicado al simulador.} \label{tab.datossimuladorpi}
\end{table}


\begin{figure}[!h]
    \centering
    \resizebox{12cm}{!}{\includegraphics{scope_simulador_pi}}
    \caption{Scope de un simulador con un control PI. El bloque PID se implanta en tiempo discreto. Se puede observar en los saltos que se producen en los ejes $dq$ cuando su referencia es nula, que las variaciones en eje $q$ son perturbaciones en eje $d$ y viceversa. Los valores del control de la simulación se recogen en la Tabla \ref{tab.datossimuladorpi}}
    \label{fig.scope_simulador_pi}
\end{figure} 


\subsection{Implantación de un Control PI en el prototipo} \label{sec.implantacioncontrolPI}

Todo control diseñado en tiempo continuo, necesariamente, se ejecuta en tiempo discreto en el microcontrolador. A continuación, se muestra en la Tabla \ref{tab.datosimplantacioncontrolPI} las condiciones en las que se implementó el control PI en el prototipo. En todas ellas, se aplico una tensión de alimentación nominal de 48V, y una corriente de referencia de eje q de 0. En todos los casos se aplica un margen de fase de $50º$, una pulsación de cruce de $500 rad/s$, y un índice de modulación de $0.5$.

\begin{table}[!h]
    \begin{center}
    \begin{tabular}{|c|c|c|} \hline\hline\hline
    \textbf{Id Referencia (pu)} & \textbf{Id Salida (pu)} & \textbf{Error Id(pu)} \\ \hline
    0.02 & 0.0253 & -0.0053 \\ \hline
    0.04&  0.04 & 0\\ \hline   
    0.08 & 0.08 & 0\\ \hline
    0.1 & 0.0933 & 0.0067\\ \hline 
    0.12 & 0.12 & 0\\ \hline 
    0.14 & 0.144 & 0.004\\ \hline 
    0.16 & 0.16 & 0\\ \hline 
    0.2 & 0.2053 & -0.00533\\ \hline 
    0.25 & 0.2533 & -0.0033\\ \hline 
    0.3 & 0.3067 & -0.0067\\ \hline 
    0.35 & 0.3333 & 0.0167\\ \hline 
    0.4 & 0.3227& 0.0773\\ \hline 
    0.5 & 0.2493 & 0.2507\\ \hline 

    \end{tabular}
    \end{center}
    \caption{Tabla con los datos de un Control PI. Implantación hardware.} \label{tab.datosimplantacioncontrolPI}
\end{table}

En la Fig. \ref{fig.excelcontrolPI} se muestra una gráfica con los datos de la Tabla \ref{tab.datosimplantacioncontrolPI}.

\begin{figure}[!h]
    \centering
    \resizebox{13cm}{!}{\includegraphics{excelcontrolPI}}
    \caption{Gráfica de error de seguimiento de referencia de corriente de eje d con un Control PI. Implantación hardware. Los valores de los ejes están escalados a un 0.1 del valor real en pu. Se observa que la bobina satura para valores superiores a 3pu (Aprox. 6.1A).}
    \label{fig.excelcontrolPI}
\end{figure} 


\clearpage

\chapter{Componentes Principales del Prototipo} \label{sec.componentesprincipales}
%%%%%%%%%%%%%%%%%%%%%%%%%%%%%%%%%%%%%%%%%%% ESTADO DEL ARTE %%%%%%%%%%%%%%%%%%%%%%%%%%%%%%%%%%%%%%%%%%%
%%%%%%%%%%%%%%%%%%%%% CONTROL DE VARIABLES TRIFASICAS %%%%%%%%%%%%%%%%%%



%%%%%%%%%%%%%%%%%%%%% MICROCONTROLADORES %%%%%%%%%%%%%%%%%%

\section{Microprocesadores y microcontroladores} \label{sec.microcontroladores}

La arquitectura de todo microcomputador puede resumirse como aparece en la Fig. \ref{fig.esquemaarquitecturaordenador}, según el modelo de Harvard\footnote{Obtenido de los apuntes de la asignatura de Microprocesadores. Último acceso: 09/01/2020. Material accesible para los alumnos matriculados en la asignatura de Microprocesadores en la Universidad Pontificia de Comillas.}. El microprocesador o CPU que puede verse, es la unidad donde se procesan, analizan y ejecutan todas las instrucciones que provienentes del resto de perifericos.

\begin{figure}[!h]
    \begin{center}
        
\resizebox{6cm}{!}{\includegraphics{esquemaarquitecturaordenador}}
\end{center}

\caption{Esquema de un computador.}

\label{fig.esquemaarquitecturaordenador}
\end{figure}

Un microcontrolador es un microcomputador o CPU comprimido todo en una sola unidad física o circuito integrado, como en la Fig. \ref{fig.esquemamicrocontrolador}. Para el desarrollo del presente proyecto, se ha tenido que discernir entre distintos microcontroladores para el correcto funcionamiento del VSC. Primero se consideró la RaspberryPi 3B+. No obstante, se decidió no utilizar éste módulo porque se buscaban unas frecuencias de conmutación mayores de las que podían conseguirse con periféricos de los que se disponía. Así mismo, las comunicaciones con los periféricos debían hacerse por i2c, lo que aumentaba la complejidad de los diagramas de Matlab/Simulink.

\begin{figure}[!h]
    \begin{center}
    \resizebox{8cm}{!}{\includegraphics{esquemamicrocontrolador}}
    \end{center}
    \caption{Esquema de un microcontrolador.}
    \label{fig.esquemamicrocontrolador}
\end{figure}

        

Será en el microcontrolador donde se correrá en tiempo real los controles diseñados. (Véase la Sección \ref{sec.controlespid}). Dichos controles se diseñarán en Matlab - Simulink.

\subsection{Familia TMS320} \label{sec.familiatms320}
Los microcontroladores TMS320 aparecieron en el mercado en el año 1982 con el microcontrolador TMS32010, de coma fija de 16-bits, de la mano del fabricante Texas Instruments. Desde entonces, la familia de DSP (del inglés \textit{Digital Signal Processor}) se ha extendido mundialmente en multitud de variantes y subfamilias \cite{quince}.

La familia TMS320 consiste en DSPs de 16-bits de coma fija, 32-bits de coma flotante y multiprocesadores de 64-bits \cite{dieciseis}. Las características de esta familia de \textit{chips}, permiten utilizar estos microcontroladores en diversos sectores,
\begin{itemize}
    \item Conjunto de instrucciones flexibles.
    \item Flexibilidad operativa.
    \item Gran velocidad de procesamiento.
    \item Arquitectura paralela.
    \item Gran eficiencia respecto a su costo.
\end{itemize}
En la Fig. \ref{fig.evolucionfamiliatms320} puede observarse las diversas familias de controladores de la familia TMS320 hasta nuestros días.

\begin{figure}[!h]
    \begin{center}
    \resizebox{8cm}{!}{\includegraphics{evolucionfamiliatms320}}
    \end{center}
    \caption{Evolución de la familia TMS320 \cite{dieciseis}.}
    \label{fig.evolucionfamiliatms320}
\end{figure}


\subsection{TMS320F28739D} \label{sec.TMS320F28739D}

El microcontrolador utlizado en el proyecto es del fabricante Texas Instruments, modelo TMS320F28379D. Analizando la nomenclatura del microcontrolador \cite{dieciseis},

\begin{itemize}
    \item TMS : Dispositivo cualificado.
    \item 320: Familia TMS320.
    \item F : CMOS con memoria flash.
    \item 28739 : generación del microprocesador.
    \item D : Doble nucleo \cite{diecisiete}.
\end{itemize}

Entre sus principales características de funcionamiento, cabría destacar las siguientes \cite{diecisiete}:
\begin{itemize}
    \item Doble núcleo de 32-bits.
    \item Reloj interno a 200MHz.
    \item Memoria interna flash de 512Kb o 1Mb.
    \item Dos osciladores internos de 10 MHz.
    \item Hasta cuatro canales de conversión ADC con hasta veinticuatro puertos diferenciales.
    \item Tres canales DAC de 12 bits.
    \item Veinticuatro puertos PWM. Dieciseis de ellos de alta resolución.
\end{itemize}

No obstante, el microcontrolador tiene limitaciones a la hora de incorporarlo en la tarjeta que lo incorpora, la LAUNCHPADXL - F28379D, como se comenta en la Sección \ref{sec.rectificacionespinout}.

\subsection{LAUNCHPADXL - F28379D} \label{sec.launchpad}

El LAUNCHPADXL - F28379D utilizado en el proyecto se muestra en las Fig. \ref{fig.launchpad_superior} y \ref{fig.launchpad_inferior}, del fabricante Texas Instruments, modelo del microcontrolador que incorpora, el TMS320F28379D, mostrado ensamblado en el LaunchPad en las Fig. \ref{fig.microf28739doverview} y \ref{fig.detallemicrof28739doverview}, dentro de la serie C2000 Microcontrollers.


\begin{figure}[!h]
    \begin{center}
        \resizebox{10cm}{!}{\includegraphics{launchpad_superior}}
        \end{center}
        \caption{Vista superior del LaunchPad F28379D.}
        \label{fig.launchpad_superior}
\end{figure}

\begin{figure}[!h]
    \begin{center}
        \resizebox{10cm}{!}{\includegraphics{launchpad_inferior}}
        \end{center}
        \caption{Vista inferior del LaunchPad F28379D.}
        \label{fig.launchpad_inferior}
\end{figure}

\begin{figure}[!h]
    \begin{center}
        \resizebox{10cm}{!}{\includegraphics{microf28739doverview}}
        \end{center}
        \caption{Tarjeta LAUNCHXL - F28379D \cite{trece}}
        \label{fig.microf28739doverview}
\end{figure}

\begin{figure}[!h]
    \begin{center}
        \resizebox{12cm}{!}{\includegraphics{detallemicrof28739doverview}}
        \end{center}
        \caption{Vista en detalle parte superior tarjeta LAUNCHXL - F28379D \cite{trece}}
        \label{fig.detallemicrof28739doverview}
\end{figure}

Los pines se localizan de tal manera que la placa del LaunchPad se acoplan directamente al VSC BOOSTXL-3PhGaNInv, y la programación se simplifica considerablemente. 

Los pines de entrada y de salida del microcontrolador se muestran en las Fig. \ref{fig.microf28379pinout}, \ref{fig.detalleizquierdomicrof28379pinout} y \ref{fig.detallederechomicrof28379pinout}. 

\begin{figure}[!h]
    \begin{center}
        \resizebox{10cm}{!}{\includegraphics{microf28379pinout}}
        \end{center}
        \caption{Tarjeta LAUNCHXL - F28379D Pinout \cite{catorce}.}
        \label{fig.microf28379pinout}
\end{figure}

\begin{figure}[!h]
    \begin{center}
        \resizebox{12cm}{!}{\includegraphics{detalleizquierdomicrof28379pinout_marcado}}
        \end{center}
        \caption{Vista de detalle de la tarjeta LAUNCHXL - F28379D Pinout lado izquierdo \cite{catorce}. Se recuadran las salidas del DAC para las que existe un bloque en el toolbox de Simulink.}
        \label{fig.detalleizquierdomicrof28379pinout}
\end{figure}
\begin{figure}[!h]
    \begin{center}
        \resizebox{12cm}{!}{\includegraphics{detallederechomicrof28379pinout_marcado}}
        \end{center}
        \caption{Vista de detalle de la tarjeta LAUNCHXL - F28379D Pinout lado derecho \cite{catorce}. Se recuadran los pines para los que no existe un bloque específico de Simulink y no pueden utilizarse con dicha librería como DAC.}
        \label{fig.detallederechomicrof28379pinout}
\end{figure}

\clearpage
\subsection{Rectificaciones PINOUT} \label{sec.rectificacionespinout}

Las Fig. \ref{fig.microf28379pinout}, \ref{fig.detalleizquierdomicrof28379pinout} y \ref{fig.detallederechomicrof28379pinout} muestran los pines genéricos para el microcontrolador TMS320F28379D, no obstante, el LAUNCHPADXL - F28379D presenta algunas diferencias en su PINOUT. Existen ciertas modificaciones que se han encontrado a la hora de desarrollar el proyecto, enumeradas a continuación:

\begin{enumerate}
    \item El pin ADCINA0 puede comportarse como ADC o como DAC-A (recuadrado en la Fig. \ref{fig.detalleizquierdomicrof28379pinout}).
    \item El pin ADCINA1 puede comportarse como ADC o como DAC-B (recuadrado en la Fig. \ref{fig.detalleizquierdomicrof28379pinout}).
    \item Los pines DAC1, DAC2, DAC3 y DAC4  no son accesibles desde un bloque de la libreria de Simulink (recuadrados en la Fig. \ref{fig.detallederechomicrof28379pinout}). 
\end{enumerate}


\subsection{Matlab y Simulink} \label{sec.matlabsimulink}

El software utilizado para el control de variables del inversor será Matlab en la versión R2019. Dentro de éste, se desarrollarán scripts para:
\begin{itemize}
\item El cálculo de variables para los controles de seguimiento.
\item Inicialización de las variables de los valores reales del modelo.
\item El cálculo de las funciones de transferencia que puedan ser relevantes para su análisis.
\item Inicialización de los protocolos de comunicación con los diferentes sensores y sondas que se hallen en el VSC.
\item La programación de los PWM que disparan los semiconductores del VSC.
\end{itemize}

Seguidamente, se hará uso del módulo de Simulink, perteneciente a Matlab, que se implementará en el microcontrolador y correrá en tiempo real en el mismo.

Existen librerias y ficheros específicos para esta placa (y otras del mismo fabricante) disponibles en una \textit{toolbox} en SIMULINK. En la Fig. \ref{fig.toolbox} se muestran los distintos bloques de Simulink para los procesadores de la serie C2000 LaunchPad. Dentro de éstos, el prototipo se desarrolla con el LaunchPad F28379D, por lo tanto, se utiliza la librería \textit{F2837xD}, desplegada en la Fig. \ref{fig.toolbox_2}. Al compilar los ficheros de Matlab y Simulink, se genera un código mediante un compilador propio del fabricante. Este código es el que se ejecuta en el LaunchPad. Para instalar la \textit{toolbox} necesaria, ver el Anexo I.
\begin{figure}[!h]
    \begin{center}
        \resizebox{12cm}{!}{\includegraphics{toolbox}}
        \end{center}
        \caption{Toolbox de los procesadores C2000 de Texas Instruments.\cite{cincuentaycinco}}
        \label{fig.toolbox}
\end{figure}

\begin{figure}[!h]
    \begin{center}
        \resizebox{12cm}{!}{\includegraphics{toolbox_2}}
        \end{center}
        \caption{Toolbox para el LaunchPad F2837xD de Texas Instruments.\cite{cincuentaycinco}}
        \label{fig.toolbox_2}
\end{figure}

%%%%%%%%%%%%%%%%%%%%% MATLAB y SIMULINK %%%%%%%%%%%%%%%%%%
\clearpage
\section{VSC BOOSTXL-3PhGaNInv} \label{sec.vscboostxl}

El VSC que se va a utilizar en el proyecto es del fabricante Texas Instruments, modelo BOOSTXL-3PhGaNInv, mostrado en las Fig. \ref{fig.vsc_superior} y \ref{fig.vsc_inferior}.

\begin{figure}[!h]
    \begin{center}
        \resizebox{12cm}{!}{\includegraphics{vsc_superior_marcado}}
        \caption{Vista superior del BOOSTXL-3PhGaNInv. Los recuadros son las resistencias cambiadas.}\label{fig.vsc_superior}
    \end{center}
\end{figure}

\begin{figure}[!h]
    \begin{center}
        \resizebox{12cm}{!}{\includegraphics{vsc_inferior}}
        \caption{Vista inferior del BOOSTXL-3PhGaNInv.}\label{fig.vsc_inferior}
    \end{center}
\end{figure}



Se trata de un convertidor CC - CA trifásico. El diagrama de funcionamiento del VSC es el mostrado en la Fig. \ref{fig.diagramaVSC}. 
\begin{figure}[!h]
    \begin{center}
        \resizebox{12cm}{!}{\includegraphics{diagramaVSC}}
        \end{center}
        \caption{Diagrama funcional del BOOSTXL-3PhGaNInv \cite{cuarentayocho}.}
        \label{fig.diagramaVSC}
\end{figure}

Este VSC cuenta con las principales características \cite{cuarentayocho}:
\begin{itemize}
    \item Tensión de alimentación entre 12V y 60V. La tensión nominal de alimentación es de 48V.
    \item Semiconductores MOSFETs de nitruro de galio.
    \item Sensores de corriente por fase con corrientes de pico entre $\pm$ 16.5A. Se han cambiado tres resistencias que eran de $5m\Omega$, por unas de $15m\Omega$, para aumentar la precisión de medida de corriente,recuadradas las tres resistencias cambiadas en la Fig. \ref{fig.vsc_superior}.
    \item Compatible con las tarjetas de evaluación de la serie C2000 MCU  (MicroController Unit).
\end{itemize}

Los componentes que aparecen en la Fig. \ref{fig.diagramaVSC} y que son necesarios explicar, son el \textit{LMG5200} y el \textit{INA240}.

El \textit{LMG5200} es un dispositivo que contiene los semiconductores que quieren controlarse. Cada dispositivo contiene dos FETS (Sección \ref{sec.semiconductores}). Cada uno de los FETS tiene la capacidad de soportar tensión de manera continua de hasta 80V y picos de tensión de hasta 100V. Así mismo, es capaz de conducir hasta 10A de corriente. Toda la información técnica puede encontrarse en \cite{cuarentayseis}. Respecto al tiempo de apagado y de encendido de los semiconductores, explicado en la Sección \ref{sec.tiempomuerto}, puede observarse el comportamiento de la activación y desactivación del semiconductor en la Fig. \ref{fig.detalledeadtimelmg5200}-a. La Fig. \ref{fig.detalledeadtimelmg5200}-b representa la zona de activación en detalle.

\begin{figure}[!h]
    \begin{center}
        \resizebox{12cm}{!}{\includegraphics{detalledeadtimelmg5200}}
        \end{center}
        \caption{a) apagado y b) encendido del LMG5200 \cite{cuarentaynueve}.}
        \label{fig.detalledeadtimelmg5200}
\end{figure}


Los semiconductores contenidos en el dispositivo son controlados mediante un driver incorporado en el mismo integrado. Puede leerse en el datasheet del LMG5200 \cite{cuarentayseis} que el driver que controla el encendido y apagado de los semiconductores, es capaz de generar un tiempo muerto (Sección \ref{sec.tiempomuerto}) de 2ns. Esta gran rapidez en los semiconductores, se debe a que están realizados en nitruro de galio, que como se describía en la Sección \ref{sec.materialessemiconductores}, son sustancialmente más rápidos en la conducción de electrones que los más comunes como los de silicio \cite{siete}.

El \textit{INA240} es un dispositivo electrónico que amplifica la corriente que circula por las fases en formato de tensión a la salida, con propiedades como el rechazo del ruido de la conmutación de PWM \cite{cuarentaysiete}. En la Fig. \ref{fig.ina240diagrama} se muestra un diagrama funcional del sistema.

\begin{figure}[!h]
    \begin{center}
        \resizebox{12cm}{!}{\includegraphics{ina240diagrama}}
        \end{center}
        \caption{Diagrama funcional del INA240 \cite{cuarentayocho}}
        \label{fig.ina240diagrama}
\end{figure}

Este dispositivo viene predeterminado con una resistencia de medida ($R_{s}$ en (\ref{eq.rs})) de $5m\Omega$. No obstante, si las corrientes que van a medirse con demasiado pequeñas, puede aumentarse la precisión de la medida modificando esta resistencia por otra de mayor valor, de manera que a igual valor de corriente por dicha resistencia, a la salida se obtenga mayor tensión. En el presente proyecto se cambia la resistencia de $5m\Omega$ por una de $15m\Omega$ (recuadradas en la Fig. \ref{fig.vsc_superior}), aumentando por tres el valor de la medida. Así mismo, el valor de la ganancia del INA240 integrado en el inversor, de 20V/V, hace que el INA240 que se utiliza sea el INA240A1, como puede verse en la Fig. \ref{fig.tablaina240}.

\begin{figure}[!h]
    \begin{center}
        \resizebox{12cm}{!}{\includegraphics{tablaina240}}
        \end{center}
        \caption{Tipos de INA240 en función de su ganancia \cite{cuarentaysiete}.}
        \label{fig.tablaina240}
\end{figure}

Sea el circuito electrónico del INA240 parametrizado en la Fig. \ref{fig.circuito_ina240}. 

\begin{figure}[!h]
    \begin{center}
        \resizebox{12cm}{!}{\includegraphics{circuito_ina240}}
        \end{center}
        \caption{Circuito electrónico del INA240. \cite{cuarentayseis}}
        \label{fig.circuito_ina240}
\end{figure}
La relación entre la entrada diferencial $V_{1} - V_{2}$ y la salida $V_{o}$ se muestra en  (\ref{eq.ft_ina240}), en la cual:  
\begin{eqnarray}
    V_{in} = V_{1} - V_{2} = \rm{IntensidadMuestreada} \cdot R_{s}  \label{eq.rs}
\end{eqnarray}

\begin{eqnarray*}
    V_A = V_S \frac{R}{R + R} = \frac{V_S}{2}
\end{eqnarray*}

\begin{eqnarray}
    V_{o} = V_1 \frac{R_2+R_4}{R_2} \frac{R_3}{R_1+R_3} - V_2 \frac{R_4}{R_2} + V_A \frac{R_1}{R_1 + R_3}\frac{R_2 + R_4}{R_2} \label{eq.ft_ina240}
\end{eqnarray}

En  (\ref{eq.ft_ina240}) puede observarse que el valor de la tensión de salida consistirá en un valor medio positivo, así como de una relación entre las resistencias que multiplican a $V_1$ y $V_2$ tal que la ganancia dependa del INA240 que se emplee, recogidos en la Fig. \ref{fig.tablaina240}.
\clearpage
\subsection{Pinout del BOOSTXL - 3PhGaNInv}

El sistema de pines del BOOSTXL - 3PhGaNInv se muestra en las Fig. \ref{fig.boostxlpinoutj1} y \ref{fig.boostxlpinoutj2}.

\begin{figure}[!h]
    \begin{center}
        \resizebox{12cm}{!}{\includegraphics{boostxlpinoutj1}}
        \end{center}
        \caption{Conexiones lado izquierdo (J1) del BOOSTXL - 3PhGaNInv \cite{cincuentayuno}.}
        \label{fig.boostxlpinoutj1}
\end{figure}

\begin{figure}[!h]
    \begin{center}
        \resizebox{12cm}{!}{\includegraphics{boostxlpinoutj2}}
        \end{center}
        \caption{Conexiones lado derecho (J2) del BOOSTXL - 3PhGaNInv \cite{cincuentayuno}.}
        \label{fig.boostxlpinoutj2}
\end{figure}


Las conexiones del BOOSTXL - 3PhGaNInv (VSC) con el LaunchPad F28379D se realizan de tal manera que coinciden:
\begin{itemize}
    \item Las medidas del VSC con los ADC del LaunchPad.
    \item Los PWM del VSC se disparan desde los puertos PWM del LaunchPad.
\end{itemize}

\clearpage
\subsection{Interconexión LaunchPad con VSC} \label{sec.conexionlaunchpadinversor}

Existen dos formas de conectar el LAUNCHPADXL F28379D con el BOOSTXL-3PhGaNInv, mostrado en las Fig. \ref{fig.conjunto_superior_1} y \ref{fig.conjunto_superior_2}.


\begin{figure}[!h]
    \begin{center}
        \resizebox{12cm}{!}{\includegraphics{conjunto_superior_1}}
        \end{center}
        \caption{Conjunto LaunchPad con VSC con la conexión en la parte superior.}
        \label{fig.conjunto_superior_1}
\end{figure}

\begin{figure}[!h]
    \begin{center}
        \resizebox{12cm}{!}{\includegraphics{conjunto_superior_2}}
        \end{center}
        \caption{Conjunto LaunchPad con VSC con la conexión en la parte inferior.}
        \label{fig.conjunto_superior_2}
\end{figure}


La Tabla \ref{tab.conexioneslaunchpadinversor}\footnote{Ha de tenerse en cuenta las limitaciones descritas en la Sección \ref{sec.rectificacionespinout}} recoge las conexiones principales entre el LaunchPad F28379D y el BOOSTXL - 3PhGaNInv cuando se conecta en la parte superior o en la parte inferior \footnote{Se toma como referencia para decir \textit{Superior} o \textit{Inferior}: Superior, el VSC inmediatamente próximo a la entrada del cable microusb; Inferior, el VSC alejado de la entrada del cable microusb.}.

\begin{table}[!h]
    \begin{center}
    \begin{tabulary}{10cm}{|C|C|C|} \hline\hline\hline
    \textbf{Launchpad F28379D Superior} &  \textbf{Launchpad F28379D Inferior} &  \textbf{BOOSTXL}  \\ \hline
    ADCIN14 & ADCIN15 & $V_{DC}$ \\ \hline
    ADCINC3 & ADCINC5 & $V_{a}$ \\ \hline
    ADCINB3 & ADCINB5 & $V_{b}$ \\ \hline
    ADCINA3 & ADCINA5 & $V_{c}$ \\ \hline
    ADCINC2 & ADCINC4 & $I_{a}$ \\ \hline 
    ADCINB2 & ADCINB4 & $I_{b}$ \\ \hline
    ADCINA2 & ADCINA4 & $I_{c}$ \\ \hline
    ADCINA0 & ADCINA1 & $V_{ref}$ \\ \hline
    PWM1A & PWM4A & PWMA (High Side) \\ \hline
    PWM1B & PWM4B & PWMA (Low Side) \\ \hline
    PWM2A & PWM5A & PWMB (High Side) \\ \hline
    PWM2B & PWM5B & PWMB (Low Side) \\ \hline
    PWM3A & PWM6A & PWMC (High Side) \\ \hline
    PWM3B & PWM6B & PWMC (Low Side) \\ \hline
    GPIO 124 & - & PWM Enable \\ \hline
   - &  GPIO 26 & PWM Enable \\ \hline
    ADCINA0 & - & DAC-A  \\ \hline
    - & ADCINA1 & DAC-B \\ \hline
    \end{tabulary}
    \end{center}
    
    \caption{Tabla con las conexiones entre el LaunchPad F28379D y el BOOSTXL-3PhGaNInv.} \label{tab.conexioneslaunchpadinversor}
    \end{table}
    
El formato utilizado en el presente proyecto es el de conectar el LaunchPad con el VSC en la parte inferior para facilitar posteriormente su encapsulado y acceso a terminales de salida para la obtención de medidas.
Las Fig. \ref{fig.conjunto_lateral} y \ref{fig.conjunto_perspectiva} muestran más vistas del ensamblado que se utiliza en el prototipo.

\begin{figure}[!h]
    \begin{center}
        \resizebox{12cm}{!}{\includegraphics{conjunto_lateral}}
        \end{center}
        \caption{Conjunto LaunchPad con VSC con la conexión en la parte inferior. Vista lateral.}
        \label{fig.conjunto_lateral}
\end{figure}

\begin{figure}[!h]
    \begin{center}
        \resizebox{12cm}{!}{\includegraphics{conjunto_perspectiva}}
        \end{center}
        \caption{Conjunto LaunchPad con VSC con la conexión en la parte inferior. Vista en perspectiva.}
        \label{fig.conjunto_perspectiva}
\end{figure}

\clearpage

El prototipo del VSC junto con el Launchpad se encapsuló de tal manera que pudieran ser accesibles todas las medidas. Así mismo, dado que se trata de un prototipo para docencia se utilizaron bananas de seguridad para evitar tener acceso a altas tensiones y/o corrientes, mostradas en las Fig. \ref{fig.foto_entradas} y \ref{fig.foto_salidas}. Los pines de la Fig. \ref{fig.foto_pines} son de tensión entre 0 y 3.3 V. Los cables entre los pines son ``tierras'' (GND, del inglés \textit{Ground}) conectados al mismo GND del LaunchPad y VSC.
También se incorporaron luces LED para monitorizar el estado del sistema, encontrandose este en lazo abierto o cerrado, el tipo de control implementado y si se encuentran acopladas las bobinas en ejes $d-q$.

\begin{figure}[!h]
    \begin{center}
        \resizebox{12cm}{!}{\includegraphics{foto_entradas}}
        \end{center}
        \caption{Encapsulado del VSC y Launchpad. Entrada de tensión continua.}
        \label{fig.foto_entradas}
\end{figure}

\begin{figure}[!h]
    \begin{center}
        \resizebox{12cm}{!}{\includegraphics{foto_salidas}}
        \end{center}
        \caption{Encapsulado del VSC y Launchpad. Salidas de tensión alterna y conexión USB. }
        \label{fig.foto_salidas}
\end{figure}


\begin{figure}[!h]
    \begin{center}
        \resizebox{12cm}{!}{\includegraphics{foto_pines_marcada}}
        \end{center}
        \caption{Encapsulado del VSC y Launchpad. Pines para osciloscopio.}
        \label{fig.foto_pines}
\end{figure}

\clearpage
%%%%%%%%%%%%%%%%%%%%% DISCUSION DE LOS RESULTADOS %%%%%%%%%%%%%%%%%%%

\section{Configuración Software del prototipo} \label{sec.discusiondelosresultados}
En esta sección se tratan los diferentes módulos de los que se compone el prototipo en el Simulink\footnote{Toda la ayuda y descripción de los bloques se encuentra disponible en la url: \url{https://es.mathworks.com/help/supportpkg/texasinstrumentsc2000/modeling.html;jsessionid=0c02b24fa4214264e460ba135ce4} . Último acceso 01/01/2020. }. Todo lo necesario para instalar en el paquete de Matlab se describe en el Anexo I.

\subsection{Módulo PWM} \label{sec.modulopwm}
El bloque de Simulink ePWM, como se llama para el LaunchPad F28379D, se muestra en la Fig. \ref{fig.moduloepwm}. Este módulo, permite configurar la activación y desactivación de los semiconductores del VSC BOOSTXL - 3PhGaNInv directamente, ya sea cada transistor por separado, o programar los transistores de una sola fase de manera que uno sea el complementario del otro. 


\begin{figure}[!h]
    \begin{center}
    \resizebox{4cm}{!}{\includegraphics{moduloepwm}}
    \caption{Bloque ePWM de Simulink. Libreria TI C2000.}
    \label{fig.moduloepwm} 
    \end{center}
\end{figure}
El PWM funciona con un contador interno, de manera que mientras el contador no llega al factor de servicio (éste en ciclos de reloj también), la salida del PWM se encuentra a nivel alto ($t_{on}$) y el resto del periodo de conmutación, la salida se encuentra a nivel bajo ($t_{off}$). 
En primer lugar, la configuración de la frecuencia de conmutación del PWM se realiza teniendo en cuenta la frecuencia del reloj interno y preescalado del microcontrolador. Esta frecuencia de conmutación, ha de introducirse en formato de periodo de conmutación, en ciclos de reloj o en segundos. Si se introduce en segundos, el microcontrolador convierte dicho valor a ciclos de reloj, de manera que para aminorar la carga computacional a la hora de compilar el código, se introducirá en forma de ciclos de reloj, atendiendo a  (\ref{eq.ciclospwm}):

\begin{equation}
    TimerPeriod = \frac{FrequencyClock}{FrequencyConm \cdot Preescale}\label{eq.ciclospwm}
\end{equation}
Donde,
\begin{itemize}
    \item TimerPeriod : Periodo de conmutación en ciclos de reloj.
    \item FrequencyClock : Frecuencia interna del reloj. La frecuencia interna del reloj del TMS320F28379D es de 200MHz \cite{diecisiete}.
    \item FrequencyConm : Frecuencia de conmutación en Hz.
    \item Preescale : Preescalado del microcontrolador. En el LaunchPad F28379D es de 4.
\end{itemize}

Así, buscando una frecuencia de conmutación de $10kHz$, sabiendo que se trabaja con un preescalado de 4:
\begin{equation*}
    TimerPeriod = \frac{200MHz}{10kHz \cdot 4} = 5000
\end{equation*}

La forma de activación descrita en la Sección \ref{sec.muestreodesenales}, puede configurarse con las opciones \textit{up count, down count o up-down count} mostrados en la Fig. \ref{fig.countpwm}. 

\begin{figure}[!h]
    \begin{center}
    \resizebox{12cm}{!}{\includegraphics{countpwm}}
    \caption{Opciones de formas de generación de PWM en el bloque ePWM de Simulink \cite{cincuentaycinco}.}
    \label{fig.countpwm} 
    \end{center}
\end{figure}

Dado que se quiere que se divida el factor de servicio (Sección \ref{sec.brevedescripcionpwm}, página \pageref{eq.factordeservicio}), en dos partes iguales en un periodo de conmutación, se utiliza la forma \textit{up-down count}. 

El cuadro principal de configuraciones, de la pestaña \textit{General}, queda como en la Fig. \ref{fig.configuracionpwm1}.


\begin{figure}[!h]
    \begin{center}
    \resizebox{12cm}{!}{\includegraphics{configuracionpwm1}}
    \caption{Bloque ePWM de Simulink. Libreria TI C2000.}
    \label{fig.configuracionpwm1} 
    \end{center}
\end{figure}

Dada la forma de implantar el VSC BOOSTXL - 3PhGaNInv con el LaunchPad F28379D en el prototipo, la configuración de los PWM se muestra en la Fig. \ref{fig.pwmenprototipo}.


\begin{figure}[!h]
    \begin{center}
    \resizebox{12cm}{!}{\includegraphics{pwmenprototipo}}
    \caption{Diagrama de la nomenclatura de los PWM en Simulink para el VSC conectado al LaunchPad en su parte inferior (Sección \ref{sec.conexionlaunchpadinversor}).}
    \label{fig.pwmenprototipo}
    \end{center}
\end{figure}

El bloque ePWM permite configurar la forma de activar y desactivar los transistores de una misma fase. Para configurar los transistores de manera que uno de los transistores (el inferior) sea el complementario del otro (el superior), debemos:

\begin{enumerate}
    \item Activar el PWM superior.
    \item No activar el PWM inferior.
    \item En la pestaña Dead Band Unit, configurar el PWM inferior de manera que éste sea el negado del superior.
\end{enumerate}

Así mismo, puede configurarse una zona de espera entre activaciones y desactivaciones de los transistores, para satisfacer el tiempo de encendido y tiempo muerto, de manera que no se cortocircuite la fuente de tensión (ver Sección \ref{sec.tiempomuerto}). Se configura de esta manera el PWM superior (ePWMxA), tal que, mientras el contador interno (CMP) no alcance el valor del factor de servicio (introducido por una señal externa, fruto ésta del control diseñado, en la pestaña \textit{Counter Compare}), la salida se encuentre a nivel alto. El resto del tiempo desde que alcanza el factor de servicio hasta que llega al valor del periodo de conmutación en ciclos de reloj, la salida se encuentra a nivel bajo. La configuración puede observarse en la Fig. \ref{fig.configuracionpwm2_1}.

\begin{figure}[!h]
    \begin{center}
    \resizebox{12cm}{!}{\includegraphics{configuracionpwm2_1}}
    \caption{Configuración ePWMxA. }
    \label{fig.configuracionpwm2_1} 
    \end{center}
\end{figure}
Así, el aspecto del bloque ePWM queda como se muestra en la Fig. \ref{fig.configuracionpwm4}.

\begin{figure}[!h]
    \begin{center}
    \resizebox{4cm}{!}{\includegraphics{configuracionpwm4}}
    \caption{Aspecto final del Bloque ePWM. Entrada del factor de servicio. }
    \label{fig.configuracionpwm4} 
    \end{center}
\end{figure}

El bloque PWM funciona mediante un contador interno. Las propiedades de disparo se configura en la pestaña \textit{Counter Compare}. En dicha pestaña, se configura el contador A (CMPA) como en la Fig. \ref{fig.configuracionpwmcmp}. Los contadores C (CMPC) y D (CMPD) han de configurarse como el contador B (CMPB).

\begin{figure}[!h]
    \begin{center}
    \resizebox{12cm}{!}{\includegraphics{configuracionpwmcmp}}
    \caption{Configuración del \textit{Counter Compare}.}
    \label{fig.configuracionpwmcmp} 
    \end{center}
\end{figure}

La configuración del ePWM inferior (ePWMxB) se realiza, en primer lugar, no activándolo desde la pestaña \textit{ePWMB}, como en la Fig. \ref{fig.configuracionpwm3_1}. En segundo accediendo a la pestaña \textit{DeadBand Unit}, se activa la opción de tiempo muerto para el transistor inferior, de manera que éste sea el inverso del PWM superior. En la ficha técnica del BOOSTXL - 3PhGaNInv se describe que el tiempo muerto de los semiconductores de nitruro de galio es de 12.5ns, por lo que, traduciendo dicho valor a ciclos de reloj, si 5000 ciclos de reloj corresponden a una frecuencia de 10kHz (0.1ms), 12.5ns corresponderían a 0.625 ciclos de reloj, y dado que el mínimo valor que puede introducirse es de 1, se deja el valor a 1. Así, el transistor inferior de una fase se configura como en las Fig. \ref{fig.configuracionpwm3_1} y \ref{fig.configuracionpwm3_2}.
\begin{figure}[!h]
    \begin{center}
    \resizebox{12cm}{!}{\includegraphics{configuracionpwm3_1}}
    \caption{Configuración ePWMxB. }
    \label{fig.configuracionpwm3_1} 
    \end{center}
\end{figure}

\begin{figure}[!h]
    \begin{center}
    \resizebox{12cm}{!}{\includegraphics{configuracionpwm3_2}}
    \caption{Configuración ePWMxB desde el \textit{DeadBand Unit}.}
    \label{fig.configuracionpwm3_2} 
    \end{center}
\end{figure}

Como se comentaba en la Sección \ref{sec.muestreodesenales} y se ve reflejado en  (\ref{eq.valorts}) de la página \pageref{eq.valorts}, el ADC se dispara cada tres periodos de conmutación del PWM. Accediendo a la pestaña \textit{Event Trigger}, configuramos el inicio de conversión del ADC cada tres eventos del PWM. Configuramos cada evento de manera que cada evento (interrupción), se active cada periodo de conmutación, véase la Fig. \ref{fig.configuracionpwmeventtrigger}.


\begin{figure}[!h]
    \begin{center}
    \resizebox{12cm}{!}{\includegraphics{configuracionpwmeventtrigger}}
    \caption{Configuración del inicio de ADC sincronizado con el ePWM.}
    \label{fig.configuracionpwmeventtrigger} 
    \end{center}
\end{figure}

El prototipo que en el presente proyecto se diseña y construye, es trifásico. Las configuraciones de los PWM son idénticas para las tres fases. 
\clearpage

Estos disparos no se hacen efectivos en la tarjeta del BOOSTXL-3PhGaNInv hasta que no se habilitan los PWM. Los PWM se activan a nivel bajo en el puerto J2-16 (Ver Fig. \ref{fig.boostxlpinoutj2} de la página \pageref{fig.boostxlpinoutj2}). Dicho pin corresponde al puerto GPIO 124 o GPIO 26 del microcontrolador, en función de los PWM que quieran utilizarse del LAUNCHPADXL - F28379D. En el diagrama de simulink se implanta el sistema de la Fig. \ref{fig.pwmenable} para activar los PWM del VSC.

\begin{figure}[!h]
    \begin{center}
    \resizebox{8cm}{!}{\includegraphics{pwmenable}}
    \caption{Configuración PWM enable del VSC.}
    \label{fig.pwmenable} 
    \end{center}
\end{figure}
Al utilzar la forma de acoplamiento descrito en la Sección \ref{sec.conexionlaunchpadinversor} de la página \pageref{sec.conexionlaunchpadinversor}, se activan los PWM del VSC desde el puerto GPIO 26, al acoplar el VSC en la parte inferior del LaunchPad.


\subsection{Obtención de Señales por puertos PWM} \label{sec.salidaporpwm}

El LaunchPad F28379D dispone de un total de doce puertos PWM. El BOOSTXL - 3PhGaNInv necesita seis de ellos para funcionar correctamente, lo que permite utilizar los seis restantes para obtener salidas de señales del sistema. El factor de servicio de los PWM será el valor de la señal en cuestión que quiere obtenerse por el puerto PWM. Dicho factor de servicio deve ser estrictamente positivo y estar comprendido entre 0 y 1.
La configuración de los puertos PWM, a diferencia de la configuración explicada anteriormente, radica en configurar el puerto ePWMxB como el puerto ePWMxA; activando el puerto ePWMxB y utilizando su contandor CMPB, como puede observarse en la Fig. \ref{fig.configuracionpwm6}. El puerto ePWMxA se configura como anteriormente, disponible en la Fig. \ref{fig.configuracionpwm2_1}.

\begin{figure}[!h]
    \begin{center}
    \resizebox{12cm}{!}{\includegraphics{configuracionpwm6}}
    \caption{Configuración pestaña \textit{ePWMB} para obtención de señales por PWM.}
    \label{fig.configuracionpwm6} 
    \end{center}
\end{figure}
El contador interno del ePWMxB seguirá un factor de servicio impuesto por la señal interna que quiere obtenerse. La pestaña \textit{Counter Compare} se configura como en la Fig. \ref{fig.configuracionpwm7} y \ref{fig.configuracionpwm8}. 

\begin{figure}[!h]
    \begin{center}
    \resizebox{12cm}{!}{\includegraphics{configuracionpwm7}}
    \caption{Configuración pestaña \textit{Counter Compare} para obtención de señales por PWM. Contadores \textit{CMPA y CMPB}.}
    \label{fig.configuracionpwm7} 
    \end{center}
\end{figure}

\begin{figure}[!h]
    \begin{center}
    \resizebox{12cm}{!}{\includegraphics{configuracionpwm8}}
    \caption{Configuración pestaña \textit{Counter Compare} para obtención de señales por PWM.Contadores \textit{CMPA y CMPB}.}
    \label{fig.configuracionpwm8} 
    \end{center}
\end{figure}

Así mismo, en la pestaña de \textit{Deadband Unit}, se desactivan todas las opciones, como puede observarse en la Fig. \ref{fig.configuracionpwm5}.


\begin{figure}[!h]
    \begin{center}
    \resizebox{12cm}{!}{\includegraphics{configuracionpwm5}}
    \caption{Configuración pestaña \textit{Deadband Unit} para obtención de señales por PWM.}
    \label{fig.configuracionpwm5} 
    \end{center}
\end{figure}

De esta manera, el bloque ePWM configurado para obtener salidas se muestra en la Fig. \ref{fig.pwmab}.

\begin{figure}[!h]
    \begin{center}
    \resizebox{4cm}{!}{\includegraphics{pwmab}}
    \caption{Aspecto final del Bloque ePWM para obtención de señales. Entradas factores de servicio.}
    \label{fig.pwmab} 
    \end{center}
\end{figure}
\clearpage
\subsection{Módulo ADC} \label{sec.moduloadc}

En el presente proyecto, los convertidores ADC muestrean las corrientes por las fases.
El bloque del ADC de la librería de Simulink para el microcontrolador TMS320F28379D es el mostrado en la Fig. \ref{fig.configuracionadc}.

\begin{figure}[!h]
    \begin{center}
    \resizebox{4cm}{!}{\includegraphics{configuracionadc}}
    \caption{Bloque ADC de Simulink. Libreria TI C2000.}
    \label{fig.configuracionadc} 
    \end{center}
\end{figure}

Una vez se ha configurado el inicio de la conversión desde el bloque PWM, se configura el bloque ADC. Éste cuenta con dos pestañas: \textit{SOC Trigger} e \textit{Input Channerls}.
En el menú \textit{SOC Trigger} seleccionamos el módulo ADC que convertirá la señal. Éste módulo ha de elegirse atendiendo a la Tabla \ref{tab.conexioneslaunchpadinversor} de la página \pageref{tab.conexioneslaunchpadinversor}. 

La resolución del ADC, dada la forma de recoger los datos, será de 12-bits. El número del inicio de los ADC (\textit{SOC Number} será diferente entre todos los ADC (en su defecto, saltará un error de compilación). La \textit{SOCx Adquisition Window} es un valor entre 0 y 15. Este valor es, proporcionalmente entre 0 y 15, la cantidad de tiempo que el \textit{Sample and Hold} del ADC se mantiene abierto leyendo la medida. El inicio del ADC es causado por una interrupción del PWM de dicha fase. De esta manera, se selecciona en la opción \textit{SOCx trigger source} la opción \textit{ePWMx ADCSOCA}  (véase la Fig. \ref{fig.configuracionpwmeventtrigger}). El ADC no desencadena ninguna acción. Así mismo, el tiempo de muestreo del ADC será el mismo que el resto de bloques del programa(\textit{Sample Time = -1}), es decir, heredado de la configuración de los parametros del Simulink. El tipo de datos utilizado y que viene por defecto es de \textit{uint16}. Finalmente, la configuración de los tres ADC necesarios para muestrear las tres corrientes de un convertidor electrónico trifásico se muestra  en la Fig. \ref{fig.configuracionadc_3}. Por último se selecciona el canal dentro del módulo ADC seleccionado en la pestaña \textit{Input Channels}, recogido en la Tabla \ref{tab.conexioneslaunchpadinversor} de la página \pageref{tab.conexioneslaunchpadinversor}.

\begin{figure}[!h]
    \begin{center}
    \resizebox{12cm}{!}{\includegraphics{configuracionadc_3}}
    \caption{Configuración de los Bloque ADC.}
    \label{fig.configuracionadc_3} 
    \end{center}
\end{figure}


\clearpage
\subsection{Módulo DAC} \label{sec.moduloDAC}

El módulo DAC del microcontrolador permite obtener por un terminal una señal analógica. El módulo DAC de Simulink es el mostrado en la Fig. \ref{fig.dac}.

\begin{figure}[!h]
    \begin{center}
    \resizebox{4cm}{!}{\includegraphics{dac}}
    \caption{Bloque DAC de Simulink. Libreria TI C2000.}
    \label{fig.dac} 
    \end{center}
\end{figure}



Aunque la configuración del DAC permita escoger entre el DAC-A, DAC-B o DAC-C, el DAC-C no puede utilizarse en el LaunchPad F28379D. Por otro lado, en la Sección \ref{sec.rectificacionespinout} se explicaba que uno de los DAC, necesariamente coincidiría con la salida $V_{ref}$ del VSC BOOSTXL-3PhGaNInv. Por ello, solo habrá disponible un solo DAC.
El DAC, por defecto, utiliza un valor medio de $1.5V$, de manera que la señal de salida sea estrictamente positiva.

\subsection{Diferencia de obtención de salidas por PWM o DAC} \label{sec.diferenciaspwmdac}

El LAUNCHPADXL F28379D dispone de un total de doce puertos PWM, pueden utilizarse estos para obtener salidas del control. El valor medio de la señal, obtenida por el puerto PWM tras ser filtrada por un filtro paso bajo, será el valor de la señal analógica (proporcional) que quiera obtenerse. Utilizando así mismo un filtro en la medida del osciloscopio de alrededor de 500Hz, puede obtenerse una señal continua muy similar a la del DAC. Y si se aplica el mismo filtro de 500 Hz al DAC, y al resto de las medidas, el desfase de las señales obtenidas será constante.

En la Fig. \ref{fig.excelpwmdac} se muestra un gráfico en el cuál se recogen los datos procedentes de sacar una corriente (de fase $a$ en este caso), por el módulo PWM y por el módulo DAC, obteniendo el valor real con una pinza amperimétrica. Puede comprobarse que el DAC muestra un valor más próximo al real que el PWM.


\begin{figure}[!h]
    \begin{center}
    \resizebox{12cm}{!}{\includegraphics{excelpwmdac}}
    \caption{Diferencia en las medidas de PWM - DAC.}
    \label{fig.excelpwmdac} 
    \end{center}
\end{figure}

No obstante, el desfase de la salida del PWM ($6 \cdot 10^{-4}s$) es menor que la salida del DAC ($7.9 \cdot 10^{-3}s$). La Fig. \ref{fig.20191218_133221_527} muestra la corriente de la fase $a$ real , medida con una pinza amperimétrica, la salida por el DAC y por un PWM. Puede comprobarse que el retardo aplicado por la salida del DAC es mucho mayor que por el PWM. Todas las medidas fueron filtradas con un filtro paso bajo de 500Hz.

\begin{figure}[!h]
    \begin{center}
    \resizebox{12cm}{!}{\includegraphics{20191218_133221_527}}
    \caption{Corriente por una fase, salida por el DAC, por el PWM y medida con pinza amperimétrica.}
    \label{fig.20191218_133221_527} 
    \end{center}
\end{figure}

\clearpage
\subsection{Obtención de un valor por un puerto DAC o PWM} \label{sec.obtenervalorpordacpwm}

Para obtener una señal por un puerto DAC o PWM es necesario realizar una serie de operaciones para asegurar que la señal de salida se muestra correctamente.
\begin{enumerate}
    \item Escalar la amplitud de la señal para que no exceda los límites de la salida. Los límites se establecen entre 0V y +3.3V. Dicho valor al que se escala puede variar en función del valor de la señal que quiere obtenerse.
    \item Convertir la señal a un valor entre 0 y 4096. Ésto se hace porque el Launchpad acepta valores de 12bits en formato \textit{uint}. Dado que el bit más significativo es el de signo, resulta un total de $2^{11} = 2048$ valores (en positivo y negativo).
    \item Establecer un valor medio para que toda la señal esté en el semieje positivo.
\end{enumerate}

En la Fig. \ref{fig.obtenervalorpordacpwm} se muestra un diagrama de bloques de simulink que se utiliza en el simulador y el software del prototipo para obtener las señales de salida.


\begin{figure}[!h]
    \begin{center}
    \resizebox{10cm}{!}{\includegraphics{obtenervalorpordacpwm}}
    \caption{Diagrama de bloques para obtener una señal por el DAC o PWM. El valor medio de 1.5V real podría variar.}
    \label{fig.obtenervalorpordacpwm} 
    \end{center}
\end{figure}

En caso de que la señal sature en los límites de saturación de la tensión de salida, en este caso por un puerto DAC, podría obtenerse una señal como la mostrada en la Fig. \ref{fig.senalsaturada} \cite{cincuentayseis}.

\begin{figure}[!h]
    \begin{center}
    \resizebox{10cm}{!}{\includegraphics{senalsaturada}}
    \caption{Señal saturada por la salida del DAC.}
    \label{fig.senalsaturada} 
    \end{center}
\end{figure}

\subsection{Equivalencias entre valor del osciloscopio y valor real} \label{sec.equivalenciasosciloscopioreal}

La obtención de valores por el osciloscopio esta directamente relacionada con los valores reales del prototipo. En las ecuaciones se muestran las ganancias a aplicar necesarias para poder obtener los valores reales a partir de las medidas del osciloscopio.


Obteniendo la señal de salida por el DAC o PWM, para obtener el valor en $pu$:
\begin{eqnarray}
 {\rm{Valor en pu}} =  \frac{\rm{Amplitud DAC en V} - \rm{ValorMedioDACMedido}\footnotemark}{0.75} \label{eq.equivalenciadacreal}
\end{eqnarray}
\footnotetext{Dicho valor podría variar en función de si está conectado el cable usb al ordenador o debido a otros factores. Se recomienda tomar el valor medio de la señal cuando la referencia es nula (y por tanto el valor medio es constante).}

Y para pasar de una referencia en pu en ejes $dq$ a valores reales en corrientes, sabiendo que la constante homotética debido a una transformación de Park invariante en potencia es $\sqrt{\frac{3}{2}}$, descrito en la Sección \ref{sec.park}:
\begin{eqnarray}
    I_{a} (pu) = I_{b} (pu)= I_{c} (pu)= \frac{I_{base}(A) \cdot Id(pu)}{ \sqrt{\frac{3}{2}}} \label{eq.equivalenciacorrientedqreal}
\end{eqnarray}
De esta manera, si se obtiene, por ejemplo, un valor de corriente de eje d de amplitud 1.8V, se tendrá una corriente en pu de:
\begin{eqnarray*}
    {Valor en pu} = \frac{1.8 - 1.5}{0.75} = 0.4 pu
\end{eqnarray*}

Y sustituyendo el resultado anterior en  (\ref{eq.equivalenciacorrientedqreal}),

\begin{eqnarray*}
    I_{a} (A) = I_{b} (A)= I_{c} (A)=  \frac{2.5(A) \cdot 0.4(pu)}{ \sqrt{\frac{3}{2}}}= 0.81649 A
\end{eqnarray*}

\clearpage
\subsection{Volcar el programa en el LaunchPad} \label{sec.volvarcodigo}

Antes de volcar el código en el LaunchPad, es necesario que la fuente de tensión de 48V no esté suministrando tensión al LaunchPad. En su defecto, no se volvará el código y saltará un error. 
Será necesario determinar un valor para el \textit{Fixed-Step} de la pestaña \textit{Solver}. Dicho valor a de ser el valor de tiempo más pequeño que se utiliza en el sistema. En el prototipo, el más pequeño es el periodo de conmutación de los PWM. Por ello, el \textit{Fixed-Step} del programa será el periodo de conmutación. En la Fig. \ref{fig.fixedstep}, el valor de la variable $T_{pwm}$ es de $0.1ms$.

\begin{figure}[!h]
    \begin{center}
    \resizebox{10cm}{!}{\includegraphics{fixedstep}}
    \caption{Configuración de la pestaña \textit{Solver} para implementación en el prototipo.}
    \label{fig.fixedstep}
    \end{center}
\end{figure}
Para poder volcar el programa en el LaunchPad, es necesario comunicar al programa cuál es el LaunchPad. Esto se realiza en las configuraciones del \textit{Configuration Properties > Hardware Implementation}. En esta ventana, se selecciona el LaunchPad. En el presente proyecto, es el \textit{TI Delfino LaunchPad F28379D}.

Se recomienda no cargar el programa en la memoria flash del LaunchPad. De esta manera, al retirar la alimentación de la fuente de tensión y del puerto USB, se detiene la ejecución y al volver a alimentar el microcontrolador, no se ejecuta el programa. En la Fig. \ref{fig.volcar_programa} se muestra como queda la configuración de la pestaña \textit{Hardware Implementation}. También se recomienda verificar que en la pestaña \textit{Code Generation} se utiliza el compilador de Texas Instruments en el cuadro de diálogo \textit{Toolchain}. Dicho compilador se ha de instalar cuando se instala el paquete \textit{Embebbed Coder Support Package for Texas Instruments C2000 Processors}, explicado en el Anexo I.

\begin{figure}[!h]
    \begin{center}
    \resizebox{10cm}{!}{\includegraphics{volcar_programa}}
    \caption{Configuración de la pestaña \textit{Hardware Implementation} para implementación en el prototipo.}
    \label{fig.volcar_programa}
    \end{center}
\end{figure}
Realizados los pasos anteriores, hacer click en `Deploy to Hardware' o `Build, Deploy \& Start' (dependiendo de la versión de Matlab instalada).

\clearpage

\chapter{Resultados Experimentales} \label{sec.osciloscopio}

El entorno de trabajo se muestra en la Fig. \ref{fig.fotolab}.
\begin{figure}[!h]
    \begin{center}
    \resizebox{10cm}{!}{\includegraphics{fotolab}}
    \caption{Entorno de trabajo.}
    \label{fig.fotolab} 
    \end{center}
\end{figure}



A continuación se muestran capturas tomadas en el osciloscopio \cite{cincuentaydos}. Todas las pruebas se realizaron con una tensión de alimentación de 48V.

La Fig. \ref{fig.20191219_112517_673} muestra la forma de onda de las corrientes trifásicas en un modo de funcionamiento normal (corriente nominal por las bobinas).
\begin{figure}[!h]
    \begin{center}
    \resizebox{10cm}{!}{\includegraphics{20191219_112517_673}}
    \caption{Corrientes trifásicas por las bobinas.  La identificación de las señales se muestra en detalle en la figura. Ver Tabla \ref{tab.20191219_112517_673}.}
    \label{fig.20191219_112517_673} 
    \end{center}
\end{figure}

\begin{table}[!h]
    \begin{center}
    \begin{tabular}{|c|c|c|c|c|c|} \hline\hline\hline
        \textbf{Canal} & \textbf{Color} & \textbf{Elemento} & \textbf{Amplitud (V)} & \textbf{Nº Espiras} & \textbf{Amperaje (A)} \\ \hline
    1 & Amarillo & $I_{a}$ & 15.35 & 5 & 3.07\\ \hline
    2 & Verde  & $I_{b}$ & 13.97 & 5 & 2.794\\ \hline   
    3 & Rosa  & $I_{c}$ & 15.35 & 5 & 3.07 \\ \hline
    \end{tabular}
    \end{center}
    \caption{Tabla con los datos de la Fig. \ref{fig.20191219_112517_673} con las corrientes trifásicas.} \label{tab.20191219_112517_673}
\end{table}

En caso de que el VSC no pueda suministrar una corriente o factor de servicio necesario para alcanzar los valores necesitados por el control, satura la integral del control y las formas de onda obtenidas son las mostradas en la Fig. \ref{fig.20191213_132600_094}.
\begin{figure}[!h]
    \begin{center}
    \resizebox{10cm}{!}{\includegraphics{20191213_132600_094}}
    \caption{Corrientes trifásicas por las bobinas saturadas.}
    \label{fig.20191213_132600_094} 
    \end{center}
\end{figure}

\clearpage
La Fig. \ref{fig.20191212_121543_752} muestra un control integral con una referencia en eje d con forma de pulsos.

\begin{figure}[!h]
    \begin{center}
    \resizebox{9cm}{!}{\includegraphics{20191212_121543_752}}
    \caption{Corrientes trifásicas por las bobinas con referencia de eje corriente de eje d en forma de pulsos. Referencia de Iq = 0. La identificación de las señales se muestra en detalle en la figura.}
    \label{fig.20191212_121543_752} 
    \end{center}
\end{figure}

\begin{comment}
\begin{table}[!h]
    \begin{minipage}{\textwidth}
    \begin{center}
    \begin{tabular}{|c|c|c|c|c|c|} \hline\hline\hline
    Canal & Color & Elemento & Amplitud (V) & Nº Espiras & Amperaje (A) \\ \hline
    1 & Amarillo & $I_{a}$ & 1.60 & 1 & 1.6\\ \hline
    2 & Verde & $I_{b}$ & 2.81 & 1 & 2.81\\ \hline   
    3 & Rosa & $I_{c}$ & 1.28 & 1 & 1.28\\ \hline
    4 & Azul Claro & $I_{d}$ & 0.705 & - & 1.153 \footnotetext[1]{Aplicando  (\ref{eq.equivalenciacorrientedqreal}) de la página \pageref{eq.equivalenciacorrientedqreal}} \\ \hline
    5 \footnotetext[2]{Solo representa cuando se aplica el escalón, no el valor del escalón en sí.} & Rojo & $I_{d}^{ref}$ & - & - & 1.25 \\ \hline
    \end{tabular}
    \end{center}
    \caption{Tabla con los datos de la Fig. \ref{fig.20191212_121543_752}) con las corrientes trifásicas.} \label{tab.20191212_121543_752}
\end{minipage}
\end{table}
\end{comment}
\clearpage

La Fig. \ref{fig.20191219_141657_180} muestra un control integral con una referencia en eje d con forma senoidal de 4Hz. La Fig. \ref{fig.20191219_141721_551} muestra en detalle la zona de cambio de referencia positiva a negativa de la Fig. \ref{fig.20191219_141657_180}.

\begin{figure}[h]
    \begin{center}
    \resizebox{9cm}{!}{\includegraphics{20191219_141657_180}}
    \caption{Corrientes trifásicas por las bobinas con referencia de eje corriente de eje d en forma senoidal de 4Hz.  La identificación de las señales se muestran en detalle en la figura. Referencia de Iq = 0.} \label{fig.20191219_141657_180}
    \end{center}
\end{figure} 

\begin{figure}[!h]
    \begin{center}
    \resizebox{9cm}{!}{\includegraphics{20191219_141721_551}}
    \caption{Detalle de las corrientes trifásicas por las bobinas con referencia de eje corriente de eje d en forma senoidal de 4Hz. La identificación de las señales se muestran en detalle en la figura. Referencia de Iq = 0.}\label{fig.20191219_141721_551}
    \end{center}
\end{figure} 
\begin{comment}

\begin{table}[!h]
    \begin{minipage}{\textwidth}
    \begin{center}
    \begin{tabular}{|c|c|c|c|c|c|} \hline\hline\hline
    Canal & Color & Elemento & Amplitud (V) & Nº Espiras & Amperaje (A) \\ \hline
    1 & Amarillo & $I_{a}$ & 3.64 & 1 & 3.64\\ \hline
    2 & Verde & $I_{b}$ & 3.23 & 1 & 3.23\\ \hline   
    3 & Rosa & $I_{c}$ & 3.65 & 1 & 3.65\\ \hline
    4 & Azul Claro & $I_{d}$ & 0.705 & - & 1.153 \footnotetext[1]{Aplicando  (\ref{eq.equivalenciacorrientedqreal}) de la página \pageref{eq.equivalenciacorrientedqreal}} \\ \hline
    5 \footnotetext[2]{Solo representa cuando se aplica el escalón, no el valor del escalón en si.} & Rojo & $I_{d}^{ref}$ & - & - & 1.25 \\ \hline
    \end{tabular}
    \end{center}
    \caption{Tabla con los datos de la Fig. \ref{fig.20191219_141657_180}) con las corrientes trifásicas con seguimiento de referencia senoidal.} \label{tab.20191219_141657_180}
\end{minipage}
\end{table}
\end{comment}
\clearpage


La Fig. \ref{fig.20191219_134238_079} muestra la respuesta temporal de seguimiento de referencia Id e Iq al mismo tiempo. Cabe destacar que las señales de $Id^{*}$,$Iq^{*}$ (referencias) e $Id$ se obtuvieron por un puerto PWM, mientras que la salida de $Id$ se obtuvo por el puerto DAC, se observa así lo comentado en la Sección \ref{sec.diferenciaspwmdac}, el puerto DAC desfasa más que el PWM, como puede observarse en la Fig. \ref{fig.20191219_134238_079}. Así mismo puede observarse la forma de onda de las corrientes cuando se aplica un flanco de subida. Las formas de las corrientes trifásicas aparecen con ruido debido a que las bobinas trabajan muy por debajo de su corriente nominal (5A).

\begin{figure}[!h]
    \begin{center}
    \resizebox{12cm}{!}{\includegraphics{20191219_134238_079}}
    \caption{Corrientes trifásicas por las bobinas. Respuesta temporal de corriente de eje d y eje q. Aplicación de las referencias de manera simultanea. La identificación de las señales se muestran en detalle en la figura.}\label{fig.20191219_134238_079}
    \end{center}
\end{figure}
\begin{comment}

\begin{table}[!h]
    \begin{minipage}{\textwidth}
    \begin{center}
    \begin{tabular}{|c|c|c|c|c|c|} \hline\hline\hline
    Canal & Color & Elemento & Amplitud (V) & Nº Espiras & Amperaje (A) \\ \hline
    1 & Amarillo & $I_{a}$ & 4.41 & 5 & 0.882 \\ \hline
    2 & Verde & $I_{b}$ & 4.16 & 5 & 0.832 \\ \hline   
    3 & Rosa & $I_{c}$ & 4.38 & 5 &  0.876\\ \hline
    4 \footnote{Medidas no disponibles}& Azul Claro  & $I_{d}$ & - & - & - \\ \hline
    5 \footnote{Medidas no disponibles}  & Rojo & $I_{d}^{ref}$ & - & - & - \\ \hline
    6 \footnote{Medidas no disponibles}& Naranja & $I_{q}$ & - & - &  -\\ \hline
    9 \footnote{Medidas no disponibles}& Azul Oscuro & $I_{q}^{ref}$ & - & - & - \\ \hline


    \end{tabular}
    \end{center}
    \caption{Tabla con los datos de la Fig. \ref{fig.20191219_134238_079}) con las corrientes trifásicas.} \label{tab.20191219_134238_079}
\end{minipage}
\end{table}
\end{comment}


La Fig. \ref{fig.20191219_140550_173} muestra un control integral al que se le aplica un flanco de subida y uno de bajada simultaneamente en ejes d y q. Aparece una forma de onda compleja en el punto de aplicación de las referencias.
 
\begin{figure}[!h]
    \begin{center}
    \resizebox{10cm}{!}{\includegraphics{20191219_140550_173}}
    \caption{Corrientes trifásicas por las bobinas. Respuesta temporal de corriente de eje d y eje q inversamente aplicado. La identificación de las señales se muestran en detalle en la figura.}
    \label{fig.20191219_140550_173} 
    \end{center}
\end{figure}
\begin{comment}

\begin{table}[!h]
    \begin{minipage}{\textwidth}
    \begin{center}
    \begin{tabular}{|c|c|c|c|c|c|} \hline\hline\hline
    Canal & Color & Elemento & Amplitud (V) & Nº Espiras & Amperaje (A) \\ \hline
    1 & Amarillo & $I_{a}$ & 4.61 & 1 & 4.61 \\ \hline
    2 & Verde & $I_{b}$ & 4.42 & 1 & 4.42 \\ \hline   
    3 & Rosa & $I_{c}$ & 4.57 & 1 &  4.57\\ \hline
    4 \footnote{Medidas no disponibles}& Azul Claro  & $I_{d}$ & - & - & - \\ \hline
    5 \footnote{Medidas no disponibles}  & Rojo & $I_{d}^{ref}$ & - & - & - \\ \hline
    6 \footnote{Medidas no disponibles}& Naranja & $I_{q}$ & - & - &  -\\ \hline
    9 \footnote{Medidas no disponibles}& Azul Oscuro & $I_{q}^{ref}$ & - & - & - \\ \hline


    \end{tabular}
    \end{center}
    \caption{Tabla con los datos de la Fig. \ref{fig.20191219_140550_173}) con las corrientes trifásicas.} \label{tab.20191219_140550_173}
\end{minipage}
\end{table}
\end{comment}


La Fig. \ref{fig.20191219_170609_664} muestra la respueta temporal de seguimiento de referencia de corriente de eje q.
 
\begin{figure}[ht]
    \begin{center}
    \resizebox{10cm}{!}{\includegraphics{20191219_170609_664}}
    \caption{Corrientes trifásicas por las bobinas. Respuesta temporal de corriente de eje q. Referencia de Id = 0. La identificación de las señales se muestran en detalle en la figura.}
    \label{fig.20191219_170609_664} 
    \end{center}
\end{figure}
\begin{comment}

\begin{table}[ht]
    \begin{minipage}{\textwidth}
    \begin{center}
    \begin{tabular}{|c|c|c|c|c|c|} \hline\hline\hline
    Canal & Color & Elemento & Amplitud (V) & Nº Espiras & Amperaje (A) \\ \hline
    1 & Amarillo & $I_{a}$ & 4.61 & 1 & 4.61 \\ \hline
    2 & Verde & $I_{b}$ & 4.69 & 1 & 4.69 \\ \hline   
    3 & Rosa & $I_{c}$ & 4.61 & 1 &  4.61\\ \hline
    6 \footnote{Medidas no disponibles}& Naranja & $I_{q}$ & - & - &  -\\ \hline
    9 \footnote{Medidas no disponibles}& Azul Oscuro & $I_{q}^{ref}$ & - & - & - \\ \hline


    \end{tabular}
    \end{center}
    \caption{Tabla con los datos de la Fig. \ref{fig.20191219_170609_664}) con las corrientes trifásicas.} \label{tab.20191219_170609_664}
\end{minipage}
\end{table}
\end{comment}

\clearpage

\newpage
\chapter{Presupuesto} \label{sec.presupuesto}

El costo del prototipo se detalla en la Tabla \ref{tab.presupuesto_materiales}. En éste no se incluyen los materiales utilizados para el encapsulado del sistema, ya que no es requerido para el correcto funcionamiento del prototipo.
\section{Presupuesto Materiales}
\begin{table}[ht]
    \begin{minipage}{\textwidth}
    \begin{center}
    \begin{tabulary}{12cm}{|C|C|C|C|} \hline\hline\hline\hline
    \textbf{Elemento} & \textbf{Cantidad ($u$)} & \textbf{Coste Unitario ($Euros/u$)}  & \textbf{Coste ($Euros$)}\\ \hline
    LaunchPad F28379D \footnote{\label{fn_pres_1}Fabricante: \textit{Texas Instruments.}}   &  1 & 34.79\footnote{\label{fn_pres_2}Distribuidor:\textit{Mouser Electronics}; \url{https://www.mouser.es/}. Último acceso: 08/01/2020} & 34.79 \\ \hline
    BOOSTXL - 3PhGaNInv \footref{fn_pres_1} & 1 & 50.45\footref{fn_pres_2} & 50.45  \\ \hline
    Bobina trifásica RTLX5. \footnote{Fabricante:\textit{Polylux.}} & 1 & 65.10\footnote{Distribuidor: \textit{Trafo Direct} ; \url{https://trafo-direct.es}. Último acceso: 08/01/2020} & 65.10\\ \hline
    Total Materiales & - & - & 150.05   \\ \hline
    \end{tabulary}
    \end{center}
    \caption{Tabla con el presupuesto de los materiales del prototipo.} \label{tab.presupuesto_materiales}
\end{minipage}
\end{table}

\section{Presupuesto Mano de Obra}

El costo de la mano de obra de desarrollo del proyecto y prototipo se detalla en la Tabla \ref{tab.presupuesto_manodeobra}.

\begin{table}[ht]
    \begin{minipage}{\textwidth}
    \begin{center}
    \begin{tabulary}{10cm}{|C|C|C|C|} \hline\hline\hline\hline
        \textbf{Tipo de mano de obra} & \textbf{Cantidad ($h$)} & \textbf{Coste Unitario ($Euros/h$)}  & \textbf{Coste ($Euros$)} \\ \hline
    Ingenieril &  150 & 36.00 & 5400  \\ \hline
    Supervisión & 30 & 90 & 2700\\ \hline
    Total Mano de Obra & - & -  & 8100\\ \hline
    \end{tabulary}
    \end{center}
    \caption{Tabla con el presupuesto de la mano de obra para la construcción del prototipo.} \label{tab.presupuesto_manodeobra}
\end{minipage}
\end{table}

\section{Presupuesto Total}

El presupuesto total se detalla en la Tabla \ref{tab.presupuesto}.

\begin{table}[ht]
    \begin{minipage}{\textwidth}
    \begin{center}
    \begin{tabulary}{12cm}{|C|C|} \hline\hline
        \textbf{Concepto} & \textbf{Coste ($Euros$)} \\ \hline
    Materiales &  150.05   \\ \hline
    Mano de obra & 8100 \\ \hline
    Total Prototipo &  8250.05\\ \hline
    \end{tabulary}
    \end{center}
    \caption{Tabla con el presupuesto de construcción del prototipo.} \label{tab.presupuesto}
\end{minipage}
\end{table}

\clearpage
% BIBLIOGRAFIA
\addcontentsline{toc}{chapter}{Referencias}  % Esto es para que aparezca ``Referencias'' en el índice
\printbibliography

\newpage

\chapter{Anexo I: Instalación del Paquete TI C2000 para Matlab}\label{sec.anexo1}


La instalación del paquete \textit{Embedded Coder Support Package for Texas Instruments C2000 Processors} necesita los siguientes requirimientos del sistema:
\begin{itemize}
    \item Tener instado Matlab R2017a o versiones posteriores, a pesar de que existen versiones de este paquete que puede ejecutarse en Matlab de versiones 2014a hasta 2019b.
    \item El sistema operativo debe ser Windows, no está disponible para MacOS ni Linux.
\end{itemize}


En la ventana emergente que aparece, buscar el paquete: \textit{ Embedded Coder Support Package for Texas Instruments C2000 Processors}. En la Fig. \ref{fig.c2000_1} se muestra la pantalla principal del paquete a instalar.

\begin{figure}[!h]
    \begin{center}
    \resizebox{10Cm}{!}{\includegraphics{c2000_1}}
    \caption{Menu Add-Ons de Matlab.}
    \label{fig.c2000_1} 
    \end{center}
\end{figure}

La instalación del paquete para \textit{C2000 Microcontrollers} necesita la previa instalación de los paquetes de Matlab:
\begin{itemize}
    \item Simulink.
    \item Embedded Coder. 
    \item Matlab Coder.
    \item Simulink Coder.
\end{itemize}


Las Fig. \ref{fig.c2000_2}, \ref{fig.c2000_3} y \ref{fig.c2000_4} muestran las pantallas siguientes en la configuración. En todas las necesarias, hacer click en \textit{Continuar} y \textit{Setup Now}.
\begin{figure}[!h]
    \begin{center}
    \resizebox{10Cm}{!}{\includegraphics{c2000_2}}
    \caption{Pantalla de Configuración 1 de la librería \textit{C2000 Processors}.}
    \label{fig.c2000_2} 
    \end{center}
\end{figure}
\begin{figure}[!h]
    \begin{center}
    \resizebox{10Cm}{!}{\includegraphics{c2000_3}}
    \caption{Pantalla de Configuración 2 de la librería \textit{C2000 Processors}.}    \label{fig.c2000_3} 
    \end{center}
\end{figure}
\begin{figure}[!h]
    \begin{center}
    \resizebox{10Cm}{!}{\includegraphics{c2000_4}}
    \caption{Pantalla de Configuración 3 de la librería \textit{C2000 Processors}.}    \label{fig.c2000_4} 
    \end{center}
\end{figure}
\clearpage

En la Fig. \ref{fig.c2000_5} se muestra la pantalla en la que se ha de elegir el procesador del LaunchPad del prototipo (o en su defecto, el que se quiera utilizar), en éste caso, se instala para el procesador \textit{TI Delfino F2837xD}.

\begin{figure}[!h]
    \begin{center}
    \resizebox{10Cm}{!}{\includegraphics{c2000_5}}
    \caption{Pantalla de Configuración 4 de la librería \textit{C2000 Processors}.}    \label{fig.c2000_5} 
    \end{center}
\end{figure}

Una vez se llega a la pantalla de la Fig. \ref{fig.c2000_6}, se han de instalar los componentes que aparecen en la ventana y se describen en la Tabla \ref{tab.c2000_requisitos}.
\begin{figure}[!h]
    \begin{center}
    \resizebox{10Cm}{!}{\includegraphics{c2000_6}}
    \caption{Pantalla de Configuración 5 de la librería \textit{C2000 Processors}.}    \label{fig.c2000_6} 
    \end{center}
\end{figure}

\begin{table}[!h]
    \begin{center}
    \begin{tabular}{|l|l|} \hline\hline\hline
    \textbf{Programa} & \textbf{Descripción} \\ \hline
    TI controlSUITE \footnote{Recuperado de: \url{http://ti.com/tool/CONTROLSUITE} . Último acceso el día 03/01/2020.} & Sotfware and Development Tools \\ \hline

    TI Code Composer Studio \footnote{Recuperado de: \url{http://software-dl.ti.com/ccs/esd/documents/ccs_downloads.html}. Último acceso el día 03/01/2020.} & TI Compiler \\ \hline
    
    TI C2000Ware \footnote{Recuperado de: \url{http://www.ti.com/tool/download/C2000WARE}. Último acceso el día 03/01/2020.} & Documentation for C2000 Microcontrollers\\ \hline

    \end{tabular}
    \end{center}
    \caption{Tabla con los componentes de Texas Instruments a instalar.} \label{tab.c2000_requisitos}
\end{table}

\clearpage

Las Fig. \ref{fig.c2000_7}, \ref{fig.c2000_8} y \ref{fig.c2000_9} muestran las páginas web en las que se descargan los programas.

\begin{figure}[!h]
    \begin{center}
    \resizebox{10Cm}{!}{\includegraphics{c2000_7}}
    \caption{Página de descarga del \textit{TI controlSUITE}.}\footnotemark   \label{fig.c2000_7} 
    \end{center}
\end{figure}

\footnotetext{Es necesario suministrar información personal para poder realizar la descarga.}

\begin{figure}[!h]
    \begin{center}
    \resizebox{10Cm}{!}{\includegraphics{c2000_8}}
    \caption{Página de descarga del \textit{TI Code Composer Studio}.}    \label{fig.c2000_8} 
    \end{center}
\end{figure}

\begin{figure}[!h]
    \begin{center}
    \resizebox{10Cm}{!}{\includegraphics{c2000_9}}
    \caption{Página de descarga del \textit{TI C2000Ware}.}    \label{fig.c2000_9} 
    \end{center}
\end{figure}

\clearpage
Por último una vez descargados los tres programas, volviendo al instalador del paquete en la ventana emergente de Matlab, ubicamos los programas instalados, mostrado en las Fig. \ref{fig.c2000_10}, \ref{fig.c2000_11} y \ref{fig.c2000_12}.

\begin{figure}[!h]
    \begin{center}
    \resizebox{10Cm}{!}{\includegraphics{c2000_10}}
    \caption{Pantalla de ubicación de la descarga del \textit{TI controlSUITE}.}    \label{fig.c2000_10} 
    \end{center}
\end{figure}
\begin{figure}[!h]
    \begin{center}
    \resizebox{10Cm}{!}{\includegraphics{c2000_11}}
    \caption{Pantalla de ubicación de la descarga del \textit{TI controlSUITE}.}    \label{fig.c2000_11} 
    \end{center}
\end{figure}
\begin{figure}[!h]
    \begin{center}
    \resizebox{10Cm}{!}{\includegraphics{c2000_12}}
    \caption{Pantalla de ubicación de la descarga del \textit{TI controlSUITE}.}    \label{fig.c2000_12} 
    \end{center}
\end{figure}
\clearpage
\newpage

\chapter{Anexo II: Interfaz del Simulador} \label{sec.anexo2}

El desarrollo de la interfaz del simulador tiene el objetivo de facilitar la modificación del entorno y condiciones de simulación por parte del usuario. Para crear la interfaz, han de seguirse los siguientes pasos:
\begin{enumerate}
    \item Crear las variables utilizadas en los cuadros de diálogo y desplegables.
    \item Crear un subsistema de todo el conjunto de simulink.
    \item Crear una máscara del subsistema.
    \item Incoporar los cuadros de diálogo y desplegables necesarios.
\end{enumerate}

En primer lugar se han de crear las variables del simulink que serán modificadas desde la interfaz que se desarrollará más adelante. Para ello, se accede en la barra superior del Simulink a la pestaña \textit{Tools > Model Explorer}. En la ventana que se abre, acceder a la opción \textit{Model Workspace*}. En ella, mostrar las variables Data Objects. En ésta, crear las variables con los nombres asignados en los bloques de simulink. Seleccionar la opción \textit{Argument} para que la interfaz pueda modificar sus valores tras la inicialización.

\begin{figure}[!h]
    \begin{center}
    \resizebox{13Cm}{!}{\includegraphics{mask5}}
    \caption{Creación de variables para la interfaz.}
    \label{fig.mask5} 
    \end{center}
\end{figure}

En segundo lugar se crea un subsistema de todo el conjunto del diagrama de simulink: Puede hacerse seleccionando todo y pulsando ctrl+G o botón derecho > \textit{Create Subsystem from Selection}. A continuación, se crea una máscara del sistema: Haciendo click con el botón derecho sobre el subsistema \textit{ Mask > Create Mask}. Una vez se ha pulsado en \textit{Create Mask}, aparecerá una ventana con la apariencia de la Fig. \ref{fig.mask1}.
\begin{figure}[!h]
    \begin{center}
    \resizebox{10Cm}{!}{\includegraphics{mask1}}
    \caption{Ventana principal máscara de simulink.}
    \label{fig.mask1} 
    \end{center}
\end{figure}

La pestaña \textit{Parameters and Dialog} permite la modificación de las pestañas, cuadros de diálogo, opciones, nombres de variables, etc, del cuadro de la interfaz. En la Fig. \ref{fig.mask2} se detallan las partes de las que se compone esta pestaña.

\begin{figure}[!h]
    \begin{center}
    \resizebox{13Cm}{!}{\includegraphics{mask2}}
    \caption{Configuraciones y descripciones de la pestaña \textit{Parameters and Dialog}.}
    \label{fig.mask2} 
    \end{center}
\end{figure}

Se ha de permitir que las variables contenidas en la interfaz se puedan modificar internamente en el simulink. Esto se realiza dejando la opción de \textit{Tuneable} en \textit{on}.
En la opción del \textit{Property Editor > Popup Options}, se modifican las opciones que contiene el cuadro de diálogo. Cada opción en una fila distinta, como se observa en la Fig. \ref{fig.mask3}. Cuando se selecciona en el cuadro de diálogo una opción, la variable asociada a dicho cuadro de diálogo toma el valor del número de fila del \textit{Popup Options}. 
\begin{figure}[!h]
    \begin{center}
    \resizebox{13Cm}{!}{\includegraphics{mask3}}
    \caption{Configuraciones y descripciones del \textit{Popup Options}.}
    \label{fig.mask3} 
    \end{center}
\end{figure}

Una vez se ha creado la interfaz de la ventana con los cuadros de diálogo, se programa el comportamiento de ésta. Esto se realiza en la pestaña \textit{Initialization}. En dicha pestaña, se programa con código en el lenguaje de Matlab, el valor que tomarán las variables del simulink cuando se han seleccionado las opciones de los cuadros de diálogo. El nombre de las variables asociadas a los cuadros de diálogo se observan en el lateral izquierdo en el apartado \textit{Dialog Variables}. En la Fig. \ref{fig.mask4} se muestra un ejemplo de programación de un cuadro de diálogo desplegable con siete opciones.

\begin{figure}[!h]
    \begin{center}
    \resizebox{13Cm}{!}{\includegraphics{mask4}}
    \caption{Configuraciones y descripciones de la pestaña \textit{Initialization}.}
    \label{fig.mask4} 
    \end{center}
\end{figure}

Una vez se ha creado la interfaz de la máscara, puede modificarse el comportamiento del simulador y hardware del prototipo desde la interfaz de la máscara.

\clearpage
\newpage
\chapter{Anexo III: Archivos}

Todos los archivos se encuentran disponibles en el siguiente repositorio de GitHub.
\begin{verbatim}
    https://github.com/JorgeSuarezPorras/ConvertidorElectronicoCC-CA
\end{verbatim}

\chapter{Anexo IV: Objetivos de Desarrollo Sostenible} \label{sec.anexo3}

La transición social hacia un desarrollo más sostenible, debe pasar necesariamente, por una mayor solidez en la implantación de las energías renovables en el sistema energético. Esta implantación debe hacerse, hoy en día, a través de convertidores electrónicos de potencia. Cualquier avance, análisis, diseño, prueba o desarrollo  que contribuya a la comprensión de estos sistemas facilitará dicha transición necesaria del modelo energético (Sección \ref{sec.motivaciondelproyecto} de motivación del proyecto).

Los objetivos de desarrollo sostenible  promovidos por la organización de Naciones Unidas (en inglés \textit{United Nations}) están (una parte de ellos) íntimamente ligados al presente proyecto. La mayor alineación con ellos podemos encontrarlos en:
\begin{enumerate}
    \item Objetivo 7: Energía asequible y no contaminante.
    \item Objetivo 9: Industria, innovación e infraestructura.
    \item Objetivo 11: Ciudades y comunidades sostenibles.
    \item Objetivo 13: Acción por el clima.
\end{enumerate}

\section*{Objetivo 7: Energía asequible y no contaminante.}


La gran mayoría de las fuentes renovables de energía eléctrica se conectan a la gran red de distribución a través de convertidores CC-CA fuentes de tensión con PWM. Dado que el presente proyecto consiste en el desarrollo de un prototipo de éste tipo, la divulgación del proyecto podría ayudar a facilitar el acceso a documentación e información para investigación de estos dispositivos. De esta manera, su desarrollo y mayor implantación en el sistema podría darse con mayor celeritud. Más aún tratandose de un dispotivo especialmente desarrollado para docencia. 

\section*{Objetivo 9: Industria innovación e infraestructura.}

Los convertidores electrónicos de potencia CC-CA fuentes de tensión con PWM se utilizan para conectar a la red eléctrica, sistemas de almacenamiento de energía tales como volantes de inercia o baterías.
Así mismo, pueden ser utilizados para optimizar procesos bastamente extendidos y relativamente poco eficientes del sistema energético actual, como por ejemplo, la utilización de PWM para optimizar las bombas de refrigeración de centrales térmicas, favoreciendo así la transición a un sistema energético más eficiente y limpio.

Así mismo, el desarrollo de nuevos materiales en ciencia e industria, ayudan a avanzar en la capacidad tecnológica de estos sistemas aumentando así su eficiencia. Materiales como el nitruro de galio que hoy en día esta siendo ampliamente investigado, y del cuál se componen los interruptores (MOSFETs) del VSC del prototipo del presente proyecto.

\section*{Objetivo 11: Ciudades y comunidades sostenibles.}

Los convertidores CC-CA CON PWM son fundamentales en la tracción eléctrica, ya sea para motores de trenes, vehículos eléctricos...etc. El presente proyecto abre la puerta a implementar los conocimientos adquiridos en el transporte limpio de personas y mercancias, favoreciendo el desarrollo de las llamas Ciudades Inteligentes (del inglés \textit{Smart Cities}).

\section*{Objetivo 13: Acción por el clima.}

La producción de dispositivos de electrónica de potencia actualmente producen emisiones contaminantes (debido a que la energía de la industria procede principalmente de fuentes no renovables), de manera que podría parecer contraproducente. No obstante, cuánto más se implante la electrónica de potencia en el sistema, menos se contaminará en el futuro, de manera que esta retroalimentación favorezca la obtención, distribución y consumo de energía no contaminante, ayudando así a mitigar los efectos del conocido cambio climático.


\chapter{Anexo V: Placa para toma de medidas} \label{sec.pcb}

Posteriormente a la entrega del proyecto de Fin de Grado, se ha desarrollado una placa de circuito impreso, o PCB (del inglés \textit{Printed Circuit Board}).
Dicha placa se ha desarrollado utilizando el software gratuito Kicad, considerando que el conexionado se realizaría posicionando el inversor \textit{BoostXL} en la parte inferior del LaunchPad, como se describe en la Sección \ref{sec.conexionlaunchpadinversor}, Fig.\ref{fig.conjunto_lateral} y Fig.\ref{fig.conjunto_perspectiva}.
\subsubsection*{Filtrado de medidas}
Las señales de salida que se seleccionan en la interfaz de simulink, se producen por PWM, es necesario filtrar la señal para poder observar el valor medio de la señal. Dado que de manera general, el osciloscopio de los laboratorios no dispone de filtros digitales para dichas medidas, se han añadido a dichas señales PWM1, PWM2 y PWM3 unos filtros de medida con la topología mostrada en las Fig.\ref{fig.filtropwm}.

\begin{figure}[!h]
    \begin{center}
        \resizebox{5cm}{!}{\includegraphics{filtropwm}}
        \end{center}
        \caption{Filtro utilizado para filtrar las medidas de salida PWM.}
        \label{fig.filtropwm}
\end{figure}
\subsubsection*{Ruido de medida}

La toma de medidas en dispositivos de electrónica de potencia puede acarrear serios problemas debido al ruido electromagnético de alta frecuencia. Convencionalmente, las sondas utilizadas en los osciloscopios de laboratorios, producen grandes ruidos en las medidas. Esto se debe a que el conector de tierra o GND (del inglés \textit{Ground}), forma un gran bucle o antena, capaz de recoger interferencias. Una forma de corregir estos ruidos de medida, es disminuir el tamaño de dicho bucle o antena. Existen distintas topologías para llevarlo a cabo. Conceptualmente se pueden observarse las diferencias en Fig.\ref{fig.ruidomedida} \footnote{Recuperado de: \url{https://www.electronicspecifier.com/products/power/oscilloscope-probing-techniques-for-measuring-power-supply-ripple}. Último acceso: 29-06-2020.}.

\begin{figure}
    \begin{minipage}[b]{.5\linewidth}
    \centering \includegraphics[width=150pt]{bigloop}
    \subcaption{Topología típica.\\ Se forma un gran bucle.}\label{fig.bigloop}
    \end{minipage}%
    \begin{minipage}[b]{.5\linewidth}
    \centering \includegraphics[width=150pt]{smallloop}
    \subcaption{Topología mejorada.\\ Se forma un bucle pequeño.}\label{fig.smallloop}
    \end{minipage}
    \caption{Topologías distintas que influyen en el ruido de medidas.} \label{fig.ruidomedida}
\end{figure}
    
El ruido presente en las medidas con las distintas topologías mostradas en la Fig.\ref{fig.ruidomedida} muestra claras diferencias, como puede observarse en la Fig.~\ref{fig.medidaruido}.

\begin{figure}
    \begin{minipage}[b]{.5\linewidth}
    \centering \includegraphics[width=150pt]{bigloopmedida}
    \subcaption{Ruido en la medida con topología \\de la Fig.~\ref{fig.bigloop}.}\label{fig.bigloopmedida}
    \end{minipage}%
    \begin{minipage}[b]{.5\linewidth}
    \centering \includegraphics[width=150pt]{smallloopmedida}
    \subcaption{Ruido en la medida con topología \\de la Fig.~\ref{fig.smallloop}.}\label{fig.smallloopmedida}
    \end{minipage}
    \caption{Ruidos en las medidas con las topologías de la Fig.~\ref{fig.ruidomedida}} \label{fig.medidaruido}
\end{figure}

El conector utilizado en el proyecto es el mostrado en la Fig. \ref{fig.conectorBNC}. Es un conector cuya medida aparece en el centro del BNC, mientras que la tierra o GND es el la parte metálica que lo rodea. Entre ambos hay un aislante plástico. La referencia de RS componentes es 546-4027.

\begin{figure}
    \begin{minipage}[b]{.5\linewidth}
    \centering \includegraphics[width=150pt]{conectorBNC1}
    \subcaption{Vista Superior.}\label{fig.conectorBNC1}
    \end{minipage}%
    \begin{minipage}[b]{.5\linewidth}
    \centering \includegraphics[width=150pt]{conectorBNC2}
    \subcaption{Vista Inferior.}\label{fig.conectorBNC2}
    \end{minipage}
    \caption{Conector BNC para sonda de medida.} \label{fig.conectorBNC}
\end{figure}

\clearpage
\subsection*{Planos}
Los planos superior e inferior de la placa se muestran a continuación,

    
\begin{figure}[!h]
    \begin{center}
        \resizebox{12cm}{!}{\includegraphics[angle=270]{PCB_FRONT.pdf}}
        \end{center}
        \caption{Plano de la parte superior de la PCB. Medidas directas.  Recuadrados los filtros de medida.}
        \label{fig.PCB_SUPERIOR}
\end{figure}

\begin{figure}[!h]
    \begin{center}
        \resizebox{12cm}{!}{\includegraphics[angle=270]{PCB_BACK.pdf}}
        \end{center}
        \caption{Plano de la parte inferior PCB.}
        \label{fig.PCB_INDERIOR}
\end{figure}

\end{document}



